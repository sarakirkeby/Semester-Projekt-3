\chapter{Kravspecifikation}

\begin{longtabu} to \linewidth{@{}l l l X[j]@{}}
    Version &    Dato &    Ansvarlig &    Beskrivelse\\[-1ex]
    \midrule
\label{version_Systemark}
\end{longtabu}


\section{Indledning}

Kort intro
\section{Ikke-funktionelle krav}
\subsection{(F)URPS+}
MoSCow er angivet i parentes ved hhv. M, S, C og/eller W, for Must, Should, Could og Won't\\
\section{Funktionelle krav}
\subsection{Aktør-kontekst diagram}
\subsection{Aktørbeskrivelse}
\begin{longtabu}to \linewidth{@{}l l X[j]@{}}
	{\large \textbf{Aktørnavn}} & {\large \textbf{Type}} & {\large \textbf{Beskrivelse}}\\ \toprule
	Bruger & Primær & Brugeren er den aktør der foretager blodtryksmålingerne. Brugeren er en person der har kendskab til systemet, samt tilladelse til at benytte systemet. Fx. sundhedsfaglig personale \\
	Tekniker & Primær & Tekniker er den aktør der foretager den årlige kalibrering af systemet. Teknikeren er en person der har kendskab til den tekniske del af systemet. Fx. medicotekniker på et sygehus\\
	
\end{longtabu}

\section{Use cases}

\begin{longtabu} to \linewidth{@{}l r X[j]@{}} %UC1%
    {\large \textbf{Use Case 1}} && \\
    \toprule
    Navn &&    Opstart system\\
    Use case ID &&    1\\
    Samtidige forløb &&    1\\
    Primær aktør &&    Brugeren\\
    Initialisere &&    Brugeren ønsker at opstarte systemet\\
    Forudsætninger &&  Patienten er koblet korrekt til systemet jf. afledning I, samt use case 2 er gennemført.\\
    Resultat &&    Systemet er kalibreret og brugeren er klar til at foretage en måling\\
    \midrule
    Hovedforløb &    1. &    Brugeren indtaster login-oplysninger og trykker på "Log-in"\--knappen. Patient oplysnings vindue åbnes \newline [1.a \textit{Forkert login}]\\
    	&			2. & Brugeren indtaster patientens CPR-nr. "??" Vindue åbnes. \newline [\textit{2.a CPR-nr er ikke gyldigt}] \\	
    	&			3. & Brugeren trykker på "nulstil"\--knappen. Systemet laver nulpunkts justering \newline [\textit{3.a Systemets nulpunktjustering er ikke korrekt}(\textbf{henvisning til kalibrering)}] \\ \midrule
    Undtagelser &    1a. & Besked omkring forkert login vises. Use Case fortsættes fra punkt 1     \\ 
    	&			2.a & Besked om forkert CPR-nr vises. Use Case fortsættes fra punkt 2 \\ 
    	&			3.a & Indikation omkring at systemet ikke er nulpunktjusteret vises. Use Case fortsættes fra punkt 3 \\ \bottomrule    
\caption{Fully dressed Use Case 1}
\label{UC1}
\end{longtabu}

\begin{longtabu} to \linewidth{@{}l r X[j]@{}} %UC2%
    {\large \textbf{Use Case 2}} && \\
    \toprule
    Navn &&    Kalibrer system\\
    Use case ID &&    2\\
    Samtidige forløb &&    1\\
    Primær aktør &&    Tekniker\\
    Initialisere &&    Tekniker ønsker at foretage den årlige kalibrering\\
    Forudsætninger && Systemet er ikke kalibreret \\
    Resultat &&    Systemet er kalibreret                     \\ \midrule
    Hovedforløb &    1. &      \\ \midrule
                
    Undtagelser &    2a. &    \\ \bottomrule
\caption{Fully dressed Use Case 2}
\label{UC2}
\end{longtabu}

\begin{longtabu} to \linewidth{@{}l r X[j]@{}} %UC3%
    {\large \textbf{Use Case 3}} && \\
    \toprule
    Navn &&    Mål blodtryk\\
    Use case ID &&    3\\
    Samtidige forløb &&    1\\
    Primær aktør &&    Brugeren\\
    Initialisere &&    Brugeren ønsker at foretage en blodtryksmåling\\
    Forudsætninger && UC1 er gennemført\\
    Resultat &&    At blodtrykket vises i kontinuerlig graf, systolisk og diastoliske blodtryk vises grafisk, samt puls vises grafisk                     \\ \midrule
    Hovedforløb &    1. &    Brugeren trykker på start måling\--kanppen \newline [1.a \textit{Blodtryk er under "..."}] \newline [1.b \textit{Blodtryk er over "..."}]\\ \midrule
                
    Undtagelser &    1a. & "Alarm" om at blodtryk er kritisk lavt. Bruger stopper alarm. UC2 fortsætter. \\  & 2.a & "Alarm" om at blodtryk er kritisk højt. Bruger stopper alarm. UC2 fortsætter. \\ \bottomrule
\caption{Fully dressed Use Case 3}
\label{UC3}
\end{longtabu}

\begin{longtabu} to \linewidth{@{}l r X[j]@{}} %UC4%
    {\large \textbf{Use Case 4}} && \\
    \toprule
    Navn &&    Afslut system\\
    Use case ID &&    4\\
    Samtidige forløb &&    1\\
    Primær aktør &&    Brugeren\\
    Initialisere &&    Brugeren ønsker at afslutte systemet og gemme måling\\
    Forudsætninger && UC1, UC2 og UC3 er gennemført\\
    Resultat &&    Blodtryksmålingens data er gemt i database og bruger er logget ud af systemet                    \\ \midrule
    Hovedforløb &    1. &    Brugeren trykker på "afslut måling"\--knappen\\[-1ex]   						 	
                &    2. &    Brugeren vælger interval der skal gemmes\newline
                	[2.a \textit{Ikke noget data til rådighed}]\\
                &    3. & Brugeren trykker på "gem og afslut"\--knappen\\ \midrule
                
    Undtagelser &    2a. & UC3 gentages    \\ \bottomrule
\caption{Fully dressed Use Case 4}
\label{UC4}
\end{longtabu}






















