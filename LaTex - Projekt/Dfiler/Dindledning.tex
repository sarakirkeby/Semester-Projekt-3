\chapter{Indledning}


\textbf{Ansvarsområde} \\
\textbf{Initialer: } \\
\\
JTH - Jeppe Tinghøj Honoré\\
TSN - Tine Skov Nielsen\\
SSK - Sara-Sofie Staub Kirkeby\\
NHL - Nicoline Hjort Larsen\\
FRM - Freja Ramsing Munk\\
MFJ - Mads Fryland Jørgensen\\

\begin{longtabu} to \linewidth{@{}  l X[j]@{}}
    Afsnit &    Ansvarlig\\[-1ex]
    \midrule
    Kravspecifikation & Alle\\
   Alt hardware-relateret & SSK, NHL, FRM\\
   Alt software-relateret & JTH, TSN, MFJ\\ 
   Accepttest & Alle\\  
\end{longtabu}

Det følgende dokument er en dokumentation, som ligger til baggrund for projektrapporten og omhandler udvikling af et blodtrykssystem.\\
I denne dokumentation er kravsspecifikationen, hovedscenariet og use cases beskrevet. De ikke-funktionelle krav er blevet beskrevet ved hjælp af vurderingsmetoden FURPS+, og er efterfulgt af en beskrivelse af de ikke-funktionelle krav. Kravene er efterfulgt af en opstilling af alle use cases, hvor de bliver beskrevet med fully-dressed use cases, med tilhørende extensions.\\[1ex]
Denne dokumentation indeholder desuden en dybdegående beskrivelse af udviklingen af software- og hardware-komponenter. Disse er beskrevet detaljeret igennem alle tre udviklingsfaser, design, implementering og test. For softwaren, er use casene blevet beskrevet ved hjælp af en domænemodel og sekvensdiagrammer. Herudover er systemet yderligere beskrevet ved hjælp af flere SysML-diagrammer og skitser af interfaces.\\[1ex]
Som en afslutning er der beskrevet en accepttest, med det formål at teste de opstillede krav. Accepttesten omfatter alle use cases og alle de ikke-funktionelle krav. Denne har til formål at understøtte systemets funktionalitet. \\
