\chapter{Acceptest}\label{Accepttest}

\begin{longtabu} to \linewidth{@{}l l l X[j]@{}}
    Version &    Dato &    Ansvarlig &    Beskrivelse\\[-1ex]
    \midrule
  
\end{longtabu}

\section{Indledning}
I accepttesten testes de krav, der er opstillet i kravspecifikationen. Det fremgår af accepttesten om hvilke scenarier og krav, der er blevet opfyldt og implementeret i systemet og hvilke, der enten ikke er testbare eller ikke er blevet implementeret. \\
Først vil der være accepttest af use casene altså de funktionelle krav efterfulgt af accepttest af de ikke funktionelle krav. \\
\\
Til test af use case 2-6 er der benyttet et signal fra PhysioNet. Signalet er et blodtrykssignal, som er vedlagt i bilag, se bilag XX.

\section{Accepttest af Use Cases}


%Use case 1 acceptest
\subsection{Use Case 1}
\textbf{Kalibrer System}

\begin{longtabu} to \linewidth{@{} c X[j] X[j] X[j] l@{}}
    ~ &	Test &    Forventet resultat &		Faktiske observationer &    Godkendt\\[-1ex]
    \midrule
    ~ &\textit{Hovedscenarie} & ~ & ~ &
    \\ \midrule
	1. & Forbind systemet til en kendt trykkilde &    Systemet er forbundet til en kendt kilde &  Systemet er forbundet til en kendt kilde   &	{\Huge \checkmark}
 \\   
 	2. & Mål det atmosfæriske tryk &    Det atmosfæriske tryk kan aflæses &  Det atmosfæriske tryk kan aflæses   &		{\Huge \checkmark}
 \\
 	3. & Aflæs spændingen målt på udgangen af det analoge filter &    At der forekommer en værdi, som efterfølgende kan bruges til lineærregression &     &		{\Huge \checkmark}
 \\
 	4. & Skriv resultatet ned og gentag punkt 1 og 2. med to nye tryk &    Der er tre værdier skrevet ned &   Nye værdier kan noteres  &		{\Huge \checkmark}
 \\
 	5. & Foretag lineærregression &   Afvigelseskoefficient er fundet &  Afvigelseskoefficient noteres   &		{\Huge \checkmark}
 \\
 	6. & Indtast afvigelses koefficient i kalibrering panel i "Log ind"\-vinduet &    Værdien er synlig i tekstboksen &  Værdien er synlig i tekstboksen   &		%{\Huge \checkmark}
\\
 	7. & Tryk på "Kalibrer"\-knappen &    Systemet er kalibreret &  Systemet viser ikke det faktiske tryk   &		{\Huge (-)}
    
 \\ \bottomrule
\caption{Accepttest af Use Case 1.}\\
\label{AT_UC1}
\end{longtabu}

%Use case 2 acceptest 
\subsection{Use Case 2}
\textbf{Opstart system}

\begin{longtabu} to \linewidth{@{} c X[j] X[j] X[j] l@{}}
	& Test	& Forventet resultat		& Faktiske observationer		& Godkendt\\[-1ex] 
	\midrule
	&\textit{Hovedscenarie} & & & 
	\\ \midrule
	1. & Indstil transducer til at måle atmosfærisk tryk & Transduceren er instillet & Atmosfærsik tryk måles & {\Huge \checkmark}
	\\
	2. & Tryk på "Nulpunksjuster"\--knap & Panel vises med intruktioner om nulpunktsjustering & Panel vises med instruktioner & {\Huge \checkmark}
	\\
	3. & Tryk på "indlæs tryk"\--knap & Et tryk vises tekstboksen & Tryk vises & {\Huge \checkmark}
	\\
	4. & Tryk på "Godkend"\--knap & Panelet lukkes & Panelet lukkes & {\Huge \checkmark}
	\\
	5. & Indtast personale-ID i brugenavnsfeltet; "1234" og indtast personlig kode i kodeordsfeltet; "fido" & Loginoplysninger bliver udfyldt & De rigtige oplysninger vises & {\Huge \checkmark}
	\\
	6. & Tryk på "Log ind"\--knappen & Log ind oplysninger er gyldige og stemmer overens med hinanden. "Blodtryks"\-vinduet vises  & Login vinduet lukkes og blodtryk vinduet vises & {\Huge \checkmark}
	\\
	\\ \midrule
	& \textit{Undtagelser} & & &\\ \midrule
	6.a & Indtast personale-ID i brugenavnsfeltet; "fido" og indtast personlig kode i kodeordsfeltet; "1234" & Nye log ind oplysninger vises & Nye log ind oplysninger vises &{\Huge \checkmark}
	\\ 
	6.b & Tryk "Log ind" & Besked om at de ikke findes vises & Besked vises & {\Huge \checkmark}
	\\
	\bottomrule

\caption{Accepttest af Use Case 2.}\\
\label{AT_UC2}	
\end{longtabu}

%Use Case 3 acceptest

\subsection{Use Case 3}
\textbf{Mål blodtryk}

\begin{longtabu} to \linewidth{@{} c X[j] X[j] X[j] l@{}}
    ~ &	Test &    Forventet resultat &		Faktiske observationer &    Godkendt\\[-1ex]
    \midrule
    ~ &\textit{Hovedscenarie} & ~ & ~ &
    \\ \midrule
    1. & Påsæt signal fra PhysioNet &    Signalet fungerer  & Signalet fungerer  &		{\Huge \checkmark} \\
    2. & Tryk på "start"\--knappen i i blodtryksvinduet &    Graf og blodtryks værdier vises på brugergrænsefladen  &  Graf og værdier vises  &		{\Huge \checkmark} 
 \\ \bottomrule
 
\caption{Accepttest af Use Case 3.}\\
\label{AT_UC3}
\end{longtabu}

%Use Case 4 acceptest

\subsection{Use Case 4}
\textbf{Filtrer signal}

\begin{longtabu} to \linewidth{@{} c X[j] X[j] X[j] l@{}}
    ~ &	Test &    Forventet resultat &		Faktiske observationer &    Godkendt\\[-1ex]
    \midrule
    ~ &\textit{Hovedscenarie} & ~ & ~ &
    \\ \midrule
    1. & Påsæt realistisk signal fra fysionet &    Signal vises i grafen &  Signalet vises med støj  &		{\Huge \checkmark}
    \\
    2. & Tryk på "Til"\--knappen under filter&   Signalet udglattes &   Signalet udglattes   &		{\Huge \checkmark}
    \\
    3. & Tryk på ”Fra”-knappen under filter & Signal udglattes ikke & Signalet udglattes ikke	& {\Huge \checkmark}
	
 \\ \bottomrule
 
\caption{Accepttest af Use Case 4.}\\
\label{AT_UC4}
\end{longtabu}

%Use Case 5 acceptest

\subsection{Use Case 5}
\textbf{Alarmer bruger}

\begin{longtabu} to \linewidth{@{} c X[j] X[j] X[j] l@{}}
    ~ &	Test &    Forventet resultat &		Faktiske observationer &    Godkendt\\[-1ex]
    \midrule
    ~ &\textit{Hovedscenarie} & ~ & ~ &
    \\ \midrule
    1. & Indstil cursor til 100 for diastolisk øvre grænse &    Der står 100 i pågældende tekst felt & Der står 100 i pågældende tekst felt    &		{\Huge \checkmark}
   	\\
   	2. & Indstil cursor til 80 for diastolisk nedre grænse &    Der står 80 i pågældende tekst felt &   Der står 80 i pågældende tekst felt  &		{\Huge \checkmark}
   	\\ 
   	3. & Indstil cursor til 100 for systolisk øvre grænse &    Der står 100 i pågældende tekst felt &  Der står 100 i pågældende tekst felt   &		{\Huge \checkmark}
   	\\
   	4. & Indstil cursor til 80 for systolisk nedre grænse &    Der står 80 i pågældende tekst felt &   Der står 80 i pågældende tekst felt  &		{\Huge \checkmark}
 	\\ 
	5. & Tryk på 'Start' knappen i blodtryksvindue &    Blodtryksmåling startes og Alarm går igang &  Blodtryksmåling startes og Alarm går igang   &		{\Huge \checkmark}
 	\\ 
 	 \midrule
 	& \textit{Undtagelser} & & & \\
 	\midrule
 	5a & Tryk på "udskyd alarm"\-knappen & Alarmen udskydes med 3 minutter & Alarmen udskydes med 3 minutter &{\Huge \checkmark}
 	\\
 	\bottomrule
\caption{Accepttest af Use Case 5.}\\
\label{AT_UC5}
\end{longtabu}

%Use Case 6 acceptest

\subsection{Use Case 6}
\textbf{Afslut system}

\begin{longtabu} to \linewidth{@{} c X[j] X[j] X[j] l@{}}
    ~ &	Test &    Forventet resultat &		Faktiske observationer &    Godkendt\\[-1ex]
    \midrule
    ~ &\textit{Hovedscenarie} & ~ & ~ &
    \\ \midrule
    1. & Tryk på "Afslut måling"\--knappen &   "Gemme"\--vinduet vises &  "Gemme"\--vinduet vises   &		{\Huge \checkmark}
   	\\
   	2. & Indtast CPR-nr "1111111118" &   CPR-nummeret synligt i pågældende tekst felt &  CPR-nummeret synligt i pågældende tekst felt   &		{\Huge \checkmark}
   	\\
   	3. & Tryk på "Gem"\--knappen &   Bekræftigelse vises &  Bekræftigelse vises    &		{\Huge \checkmark}
   	\\ 
   	3. & Tryk på "Ok"\-knappen &  "Gemme"\--vindue og "Blodtryks"--vinduet lukkes. "Login"\--vinduet vises  &   "Gemme"\--vindue og "Blodtryks"--vinduet lukkes. "Login"\--vinduet vises  &		{\Huge \checkmark}
   	\\ \midrule
   	
	~ &\textit{Undtagelser} & ~ & ~ & 
	\\ \midrule
	1.a. & Tryk på "annuller"\--knap & "Gemme"\--vinduet lukkes og "Blodtryk"\-- vinduet vises &	"Gemme"\--vinduet lukkes og "Blodtryk"\-- vinduet vises	&{\Huge \checkmark}
	\\
	2.a. & Indtast CPR-nummeret "1234567890" & Beskeden "CPR ikke gyldigt." vises\ & 	Beskeden "CPR ikke gyldigt." vises	& {\Huge \checkmark}	
 \\ \bottomrule
 
\caption{Accepttest af Use Case 7.}\\
\label{AT_UC7}
\end{longtabu}


\section{Accepttest af ikke-funktionelle krav}

\begin{longtabu} to \linewidth{@{} c X[j] X[j] X[j] X[j] l@{}}
	& Ikke-funktionelt krav & Test/handling & Forventet resultat & Faktiske observationer & Godkendt
	\\[-1ex] \midrule
	&  \textit{Functionality} &  &  & & \\ \midrule
	1. & Brugeren skal kunne starte en ny måling indenfor 30 sekunder efter opstart & Start programmet, hvorefter der vha. stopur måles opstartstiden & At programmet er opstartet og ny måling er igang efter 30 sekunder & 15 sekunder & {\Huge \checkmark}
	\\ \midrule
	2. & Systemet skal kunne forstærke signalet med det indbyggede analoge antialiaserings filter med en båndbredde på 50 Hz & Start systemet & At signalet er forstærket & Egentlige test ligger i integrationstest, se afsnit \ref{ModulHard}  & {\Huge (\checkmark)}\\ \midrule
	3. & Programmet skal kunne vise blodtrykssignalet kontiunert & Tryk på "Start måling"\--knap & At blodtrykssignalet er vist kontinuert på brugergrænsefladen & Blodtrykssignalet er vist kontinuert på brugergrænsefladen & {\Huge \checkmark}\\ \midrule
	4. & Programmet skal programmeres i C\# & Start programmet & At koden er i C\# & Programmet er i C\# & {\Huge \checkmark} \\ \midrule

	5. & Programmet bør kunne måle og afbillede puls & Tryk på "Start måling" & At pulsen er afbilledet på brugergrænseflade & Pulsen af afbilledet & {\Huge \checkmark} \\ \midrule
	
	& \textit{Usability} &  &  & & \\ \midrule
	1. & Blodtrykstallene der udskrives på brugergrænseflade er røde & Tryk "Start måling"\--knap & At blodtrykstallene er røde & Blodtrykstallene er røde & {\Huge \checkmark} \\ \midrule
	2. & Pulsmålingen skal udskrives på brugergrænsefladen med grønne tal & Tryk "Start måling"\--knap & At pulsen vises med grønne tal & Pulsen vises med grønne tal&  {\Huge \checkmark} \\ \midrule
	3. & Brugergrænseflade lever op til figuren udarbejdet i design afsnittet, se afsnit \ref{GUI} & Opstart program og log ind & At brugergrænseflade indeholder samtlige funktioner som på figuren & Brugergrænseflade indeholder samtlige funktioner som på figuren & {\Huge \checkmark} \\ \midrule
	& \textit{Reliability} & & & & \\ \midrule
	1. & Systemet skal kunne køre uden fejl i et år & Start system op og vent et år & At programmet efter et år kører fejlfrit & Kan ikke testes & \\ \midrule
	2. & Systemet skal have en "mean time to restore" på højest 24 timer & Start systemet og herefter genstart, hvor der tages tid med et stopur & At programmet er klar igen inden for 24 timer & Kan ikke testes & \\ \midrule
	& \textit{Performance} & & & & \\ \midrule
	1. & Systemet bør kunne gemme data på 5 sekunder +/- 10\% & Tryk på "Gem og afslut"\--knap og tag tid med stopur & At data er inden for 5 sekunder & Afhænger af hvor stor måling er & {\Huge (\checkmark)}\\ \midrule
	& \textit{Supportability} & & & & \\ \midrule
	1. & Softwaren er opbygget af trelagsmodellen & Kig i koden efter data-lag, logik-lag og GUI-lag & At koden indeholder et data-lag, et logik-lag og et GUI-lag & Koden indeholder et data-lag, et logik-lag og et GUI-lag & {\Huge \checkmark}\\ \bottomrule
\caption{Accepttest af Ikke-funktionelle krav}
\end{longtabu}

