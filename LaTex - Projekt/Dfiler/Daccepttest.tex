\chapter{Acceptest}

\begin{longtabu} to \linewidth{@{}l l l X[j]@{}}
    Version &    Dato &    Ansvarlig &    Beskrivelse\\[-1ex]
    \midrule
    1.0 &    18-03-2015 &    LSB, AJF og MFJ &    Påbegyndt tilrettelse i forhold til den valgte sygdom, Atrieflimren.\\
    1.1 &    26-03-2015 &    LSB, AJF, MFJ og CAA &    AT færdigskrevet og klar til review\\
    2.0 &    09-04-2015 &    LSB, AJF, MFJ og CAA &    Rettet i forhold til review-kommentarer\\
    3.0 &    19-05-2015 &    LSB og MFJ &    Rettet til, så accepttest udførelse kan foretages\\
    3.1	&	20-05-2015	&	ALLE			&	Resultat af udførelse af accepttest indskrevet. Dertil også skrevet fejlrapport\\
\label{version_Systemark}
\end{longtabu}

\section{Accepttest af Use Cases}


%Use case 1 acceptest
\subsection{Use Case 1}
\textbf{Log ind}

\begin{longtabu} to \linewidth{@{} c X[j] X[j] X[j] l@{}}
    ~ &	Test &    Forventet resultat &		Faktiske observationer &    Godkendt\\[-1ex]
    \midrule
    ~ &\textit{Hovedscenarie} & ~ & ~ &
    \\ \midrule
    1. &Indtast username "moh04" samt password; 1234 &   Username- og passwordboks bliver udfyldt  &   Som forventet  &		{\Huge \checkmark}
    \\
    2. &Tryk på "Login"\--knappen  &    Login bliver godkendt. Login-vinduet lukkes ned mens CPR-vinduet åbnes  &    Som forventet &		{\Huge \checkmark}
	\\ \midrule
	~ &\textit{Exentions} & ~ & ~ & 
	\\ \midrule	
    2a. &	Username eller password er forkert &    Besked vises på skærmen med tekst, der informerer om, at brugernavn eller password er forkert  &   Som forventet  &		{\Huge \checkmark}
 \\ \bottomrule
 
\caption{Accepttest af Use Case 1.}\\
\label{AT_UC1}
\end{longtabu}

%Use case 2 acceptest 
\subsection{Use Case 2}
\textbf{Vis EKG}

\begin{longtabu} to \linewidth{@{} c X[j] X[j] X[j] l@{}}
	& Test	& Forventet resultat		& Faktiske observationer		& Godkendt\\[-1ex] 
	\midrule
	&\textit{Hovedscenarie} & & & 
	\\ \midrule
	1. &Indtast virtuel patients CPR-nummer; 123456-7890 & CPR-nummerboks bliver udfyldt & Som forventet& {\Huge \checkmark}
	\\
	2. & Tryk på "Ok"\--knappen & CPR er gyldig. CPR-vinduet lukkes ned mens EKG-vinduet åbnes & Som forventet& {\Huge \checkmark}
	\\
	3. & Tryk på "Start ny måling" & Målingen startes i EKG-vinduet & Som forventet & {\Huge \checkmark}
	\\
	4. & EKG-data illustreres på en graf & En analyserebar graf fremvises i EKG-vinduet & Graf vises efter ca. 20 sekunder & {\Huge \checkmark}
	\\ \midrule
	2.a & CPR-nummeret findes ikke. Besked vises med tekst, der informerer om, at CPR-nummeret ikke er gyldigt & Nyt CPR-nummer indtastes & Som forventet & {\Huge \checkmark}
	\\ \bottomrule

\caption{Accepttest af Use Case 2.}\\
\label{AT_UC2}	
\end{longtabu}

%Use Case 3 acceptest

\subsection{Use Case 3}
\textbf{Evaluer EKG}

\begin{longtabu} to \linewidth{@{} c X[j] X[j] X[j] l@{}}
    ~ &	Test &    Forventet resultat &		Faktiske observationer &    Godkendt\\[-1ex]
    \midrule
    ~ &\textit{Hovedscenarie} & ~ & ~ &
    \\ \midrule
    1. & Validere programmets analyse af EKG-signalet &    Det er muligt at se små fluktuationer, som kan aflæses på EKG-grafen  &    Grafen er analyserbar, dog er det ikke de små fluktuationer som analyseres, se fejlrapport i bilag&		{\Huge (\checkmark)}
    \\
    2. &Stil diagnosen atrieflimmer	 &    Atrieflimmer kan aflæses ud fra EKG-grafen  &     Som forventet &		{\Huge \checkmark}
	\\ \midrule
	~ &\textit{Exentions} & ~ & ~ & 
	\\ \midrule	
    2a. &	Atriefrekvensen er ikke i intervallet 220-300 pr. minut &    Det er ikke muligt at diagnosticere atrieflimmer ud fra EKG-grafen   &   Hvis ikke atrieflimmer er diagnostiseret, vises besked om sundt EKG. Dog skyldes det ikke atriefrekvensen, se fejlrapport i bilag &		{\Huge (\checkmark)}
 \\ \bottomrule
 
\caption{Accepttest af Use Case 3.}\\
\label{AT_UC3}
\end{longtabu}

%Use Case 4 acceptest

\subsection{Use Case 4}
\textbf{Gem EKG}

\begin{longtabu} to \linewidth{@{} c X[j] X[j] X[j] l@{}}
    ~ &	Test &    Forventet resultat &		Faktiske observationer &    Godkendt\\[-1ex]
    \midrule
    ~ &\textit{Hovedscenarie} & ~ & ~ &
    \\ \midrule
    1. &Tryk på "Gem-ny-måling"\--knappen. &    Messagebox kommer frem med besked om at målingen er gemt  &    Som forventet &		{\Huge \checkmark}
    \\
    2. & Tryk på "Ok"\--knappen	 &   Målingen er gemt, vinduet lukkes og EKG-vinduet vises igen &     Som forventet &		{\Huge \checkmark}
    \\
	\\ \midrule
	~ &\textit{Exentions} & ~ & ~ & 
	\\ \midrule	
 \\ \bottomrule
 
\caption{Accepttest af Use Case 4.}\\
\label{AT_UC4}
\end{longtabu}

%Use Case 5 acceptest

\subsection{Use Case 5}
\textbf{Log ud}

\begin{longtabu} to \linewidth{@{} c X[j] X[j] X[j] l@{}}
    ~ &	Test &    Forventet resultat &		Faktiske observationer &    Godkendt\\[-1ex]
    \midrule
    ~ &\textit{Hovedscenarie} & ~ & ~ &
    \\ \midrule
    1. & Tryk på "log ud"\--knappen &    EKG-vinduet lukkes ned, mens login-vinduet fremkommer &    Som forventet &		{\Huge \checkmark}
   	\\ \midrule
	~ &\textit{Exentions} & ~ & ~ & 
	\\ \midrule	
 \\ \bottomrule
 
\caption{Accepttest af Use Case 5.}\\
\label{AT_UC5}
\end{longtabu}

\section{Accepttest af ikke-funktionelle krav}

\begin{longtabu} to \linewidth{@{} c X[j] X[j] X[j] X[j] l@{}}
	& Ikke-funktionelt krav & Test/handling & Forventet resultat & Faktiske observationer & Godkendt
	\\[-1ex] \midrule
	&  \textit{Usability} &  &  & & \\ \midrule
	& Brugeren skal kunne starte en default-måling maksimalt 20 sekunder efter opstart af program & Start programmet, hvorefter der vha. stopur måles opstartstiden & At programmet er startet op indenfor 20 sekunder & Programmet er startet op efter 14 sekunder & {\Huge \checkmark}\\ \midrule
	& Login-vinduet skal indholde en "login"\--knap til at logge på og få vist EKG-vinduet & "login"\--knappen er synlig i GUI, og ved tryk på knappen vises EKG-vinduet & At EKG-vinduet vises & Som forventet & {\Huge \checkmark}\\ \midrule 
	& EKG-vinduet skal indeholde en "start"\--knap til at igangsætte målingen & "Start"\--knappen er synlig i GUI, og ved tryk på knap igangsættes målingen & At målingen igangsættes & Som forventet& {\Huge \checkmark}\\ \midrule
	& EKG-vinduet skal indeholde en "gem"\--knap til at gemme målingerne & "Gem"\--knappen er synlig i GUI, og ved tryk på knappen gemmes måling i database & Messageboks vises på skærmen med teksten "Måling er gemt" og kan findes i databasen & Som forventet & {\Huge \checkmark}\\ \midrule
	& EKG-vinduet skal indeholde en "log ud"\--knap til at logge ud & "log ud"\--knappen er synlig i GUI, og ved tryk på knap lukkes EKG-vinduet og login-vinduet vises & Login-vinduet vises & Som forventet & {\Huge \checkmark}\\ \midrule
	& \textit{Reliability} & & & & \\ \midrule
	& Systemet skal have en effektiv MTBF på 20 minutter og MTTR på 1 minut & Køre programmet i 20 minutter. Genstart derefter programmet, hvor der tages tid med et stopur & Programmet har kørt i 20 minutter og genstartes indenfor 1 minut  & Som forventet & {\Huge \checkmark}\\ \midrule
	& \textit{Performance} & & & & \\ \midrule
	& Der skal vises en EKG-graf i interfacet, hvor spænding vises op ad y-aksen (-1V til 1V) og tiden på x-aksen & Gennemfør en måling & At spændingen for EKG-signalet er op ad y-aksen, samt tiden hen ad x-aksen & Spændingen er op ad y aksen og tiden i sekunder hen ad x-aksen. Dog er intervallet ikke -1V til 1V, se fejlrapport i bilag & {\Huge \ding{55}}\\ \midrule
	& Det skal være muligt at kunne scrolle igennem målingerne hen ad x-aksen & Der gennemføres en måling hvorefter der scrolles hen ad x-aksen & At der ved scrolling kan ses forskellige dele af EKG-signalet hen ad x-aksen & & {\Huge \checkmark}\\ \midrule
	& \textit{Supportability} & & & & \\ \midrule
	& Softwaren er opbygget af trelagsmodellen & Kig i koden efter data-lag, logik-lag og GUI-lag & At koden indeholder et data-lag, et logik-lag og et GUI-lag & Som forventet & {\Huge \checkmark}\\ \bottomrule
\caption{Accepttest af Ikke-funktionelle krav}
\end{longtabu}

