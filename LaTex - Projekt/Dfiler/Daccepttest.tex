\chapter{Acceptest}

\begin{longtabu} to \linewidth{@{}l l l X[j]@{}}
    Version &    Dato &    Ansvarlig &    Beskrivelse\\[-1ex]
    \midrule
  
\end{longtabu}

\section{Accepttest af Use Cases}


%Use case 1 acceptest
\subsection{Use Case 1}
\textbf{Kalibrer System}

\begin{longtabu} to \linewidth{@{} c X[j] X[j] X[j] l@{}}
    ~ &	Test &    Forventet resultat &		Faktiske observationer &    Godkendt\\[-1ex]
    \midrule
    ~ &\textit{Hovedscenarie} & ~ & ~ &
    \\ \midrule
    1. & Systemet kalibreres &   At systemet er kalibreret  &    &		%{\Huge \checkmark}
    
 \\ \bottomrule
\caption{Accepttest af Use Case 1.}\\
\label{AT_UC1}
\end{longtabu}

%Use case 2 acceptest 
\subsection{Use Case 2}
\textbf{Opstart system}

\begin{longtabu} to \linewidth{@{} c X[j] X[j] X[j] l@{}}
	& Test	& Forventet resultat		& Faktiske observationer		& Godkendt\\[-1ex] 
	\midrule
	&\textit{Hovedscenarie} & & & 
	\\ \midrule
	1. & Indtast personale-ID i brugenavnsfeltet; "1234" og indtast personlig kode i kodeordsfeltet; "fido" & Loginoplysninger bliver udfyldt & & %{\Huge \checkmark}
	\\
	2. & Tryk på "Log ind"\--knappen & Log ind oplysninger er gyldige og stemmer overens med hinanden. Teksfelter til log ind skjules og "Nulstil"\--knappen vises & & %{\Huge \checkmark}
	\\
	3. & Tryk på "Nulstil"\--knap & Besked om at systemet er nulpunktsjusteret vises i "Ok"\--vinduet som åbnes & & %{\Huge \checkmark}
	\\
	4. & Tryk på "Ok"\--knappen & "Log ind"\--vinduet og "Ok"\-- vinduet lukkes og "Blodtryk"\--vinduet åbnes &  & %{\Huge \checkmark}
	\\ \midrule
	2.a & Log ind oplysninger findes ikke i databasen. Besked vises med tekst, der informerer herom & Nye log ind oplysninger indtastes &  & %{\Huge \checkmark}
	\\ \bottomrule

\caption{Accepttest af Use Case 2.}\\
\label{AT_UC2}	
\end{longtabu}

%Use Case 3 acceptest

\subsection{Use Case 3}
\textbf{Mål blodtryk}

\begin{longtabu} to \linewidth{@{} c X[j] X[j] X[j] l@{}}
    ~ &	Test &    Forventet resultat &		Faktiske observationer &    Godkendt\\[-1ex]
    \midrule
    ~ &\textit{Hovedscenarie} & ~ & ~ &
    \\ \midrule
    1. & Tryk på "start-måling"\--knappen &    Graf og blodtryks værdier vises på brugergrænsefladen  &    &		%{\Huge (\checkmark)}
	\\ \midrule
	~ &\textit{Exentions} & ~ & ~ & 
	\\ \midrule	
    1.a. &	Indtast tallet 0 som ned grænse for både systolisk og diastolisk blodtryk og 400 som øvregrænse for systolisk og diastolisk blodtryk  &    Alarm i form af lyd går i gang og og der indikeres med pil (op/ned) ud fra systolisk og/eller diastolisk blodtryk  &    &		%{\Huge (\checkmark)}
 \\ \bottomrule
 
\caption{Accepttest af Use Case 3.}\\
\label{AT_UC3}
\end{longtabu}

%Use Case 4 acceptest

\subsection{Use Case 4}
\textbf{Filtrer signal}

\begin{longtabu} to \linewidth{@{} c X[j] X[j] X[j] l@{}}
    ~ &	Test &    Forventet resultat &		Faktiske observationer &    Godkendt\\[-1ex]
    \midrule
    ~ &\textit{Hovedscenarie} & ~ & ~ &
    \\ \midrule
    1. & Påsæt sinus signal med frekvens XXHz(højfrekvent) &    Sinus signal vises på grafen  &    &		%{\Huge \checkmark}
    \\
    2. & Tryk på ”filtrer signal”\--knappen &   Signalet udglattes &      &		%{\Huge \checkmark}
    \\
    3. & Påsæt sinus signal med frekvens XXHz(lavfrekvent) &Sinus signal vises på grafen & 	& %{\Huge \checkmark} 
    \\
    4. & Tryk på ”filtrer signal”-knap & Sinus-signal udglattes ikke &	& %{\Huge \checkmark}
	
 \\ \bottomrule
 
\caption{Accepttest af Use Case 4.}\\
\label{AT_UC4}
\end{longtabu}

%Use Case 5 acceptest

\subsection{Use Case 5}
\textbf{Juster grænseværdier}

\begin{longtabu} to \linewidth{@{} c X[j] X[j] X[j] l@{}}
    ~ &	Test &    Forventet resultat &		Faktiske observationer &    Godkendt\\[-1ex]
    \midrule
    ~ &\textit{Hovedscenarie} & ~ & ~ &
    \\ \midrule
    1. & Indtast 140 for diastolisk øvre grænse &    Der står 140 i pågældende tekst felt &     &		%{\Huge \checkmark}
   	\\
   	2. & Indtast 100 for diastolisk nedre grænse &    Der står 100 i pågældende tekst felt &     &		%{\Huge \checkmark}
   	\\ 
   	3. & Indtast 120 for systolisk øvre grænse &    Der står 120 i pågældende tekst felt &     &		%{\Huge \checkmark}
   	\\
   	4. & Indtast 80 for systolisk øvre grænse &    Der står 80 i pågældende tekst felt &     &		%{\Huge \checkmark}
 \\ \bottomrule
 
\caption{Accepttest af Use Case 5.}\\
\label{AT_UC5}
\end{longtabu}

%Use Case 6 acceptest

\subsection{Use Case 6}
\textbf{Udskyd alarm}

\begin{longtabu} to \linewidth{@{} c X[j] X[j] X[j] l@{}}
    ~ &	Test &    Forventet resultat &		Faktiske observationer &    Godkendt\\[-1ex]
    \midrule
    ~ &\textit{Hovedscenarie} & ~ & ~ &
    \\ \midrule
    1. & Tryk på "Udskyd alarm"\--knappen &    Alarmen stopper og starter igen 60 sekunder senere &     &		%{\Huge \checkmark}
 \\ \bottomrule
 
\caption{Accepttest af Use Case 6.}\\
\label{AT_UC6}
\end{longtabu}

%Use Case 7 acceptest

\subsection{Use Case 7}
\textbf{Afslut system}

\begin{longtabu} to \linewidth{@{} c X[j] X[j] X[j] l@{}}
    ~ &	Test &    Forventet resultat &		Faktiske observationer &    Godkendt\\[-1ex]
    \midrule
    ~ &\textit{Hovedscenarie} & ~ & ~ &
    \\ \midrule
    1. & Tryk på "Afslut måling"\--knappen &   "Gemme"\--vinduet vises &     &		%{\Huge \checkmark}
   	\\
   	2. & Indtast CPR-nr "1111111111" &   CPR-nummeret synligt i pågældende tekst felt &     &		%{\Huge \checkmark}
   	\\
   	3. & Tryk på "Gem og afslut"\--knappen &   "Gemme"\--vindue og "Blodtryks"--vinduet lukkes. "Login"\--vinduet vises?\ &     &		%{\Huge \checkmark}
   	\\ \midrule
	~ &\textit{Exentions} & ~ & ~ & 
	\\ \midrule
	1.a. & Tryk på "anuller"\--knap & "Gemme"\--vinduet lukkes og "Blodtryk"\-- vinduet vises &		&%{\Huge \checkmark}
	\\
	2.a. & Indtast CPR-nummeret "1234567890" & Besked om at CPR-nummer ikke er gyldigt vises & 		& %{\Huge \checkmark}	
 \\ \bottomrule
 
\caption{Accepttest af Use Case 7.}\\
\label{AT_UC7}
\end{longtabu}


\section{Accepttest af ikke-funktionelle krav}

\begin{longtabu} to \linewidth{@{} c X[j] X[j] X[j] X[j] l@{}}
	& Ikke-funktionelt krav & Test/handling & Forventet resultat & Faktiske observationer & Godkendt
	\\[-1ex] \midrule
	&  \textit{Usability} &  &  & & \\ \midrule
	& Brugeren skal kunne starte en default-måling maksimalt 20 sekunder efter opstart af program & Start programmet, hvorefter der vha. stopur måles opstartstiden & At programmet er startet op indenfor 20 sekunder & Programmet er startet op efter 14 sekunder & {\Huge \checkmark}\\ \midrule
	& Login-vinduet skal indholde en "login"\--knap til at logge på og få vist EKG-vinduet & "login"\--knappen er synlig i GUI, og ved tryk på knappen vises EKG-vinduet & At EKG-vinduet vises & Som forventet & {\Huge \checkmark}\\ \midrule 
	& EKG-vinduet skal indeholde en "start"\--knap til at igangsætte målingen & "Start"\--knappen er synlig i GUI, og ved tryk på knap igangsættes målingen & At målingen igangsættes & Som forventet& {\Huge \checkmark}\\ \midrule
	& EKG-vinduet skal indeholde en "gem"\--knap til at gemme målingerne & "Gem"\--knappen er synlig i GUI, og ved tryk på knappen gemmes måling i database & Messageboks vises på skærmen med teksten "Måling er gemt" og kan findes i databasen & Som forventet & {\Huge \checkmark}\\ \midrule
	& EKG-vinduet skal indeholde en "log ud"\--knap til at logge ud & "log ud"\--knappen er synlig i GUI, og ved tryk på knap lukkes EKG-vinduet og login-vinduet vises & Login-vinduet vises & Som forventet & {\Huge \checkmark}\\ \midrule
	& \textit{Reliability} & & & & \\ \midrule
	& Systemet skal have en effektiv MTBF på 20 minutter og MTTR på 1 minut & Køre programmet i 20 minutter. Genstart derefter programmet, hvor der tages tid med et stopur & Programmet har kørt i 20 minutter og genstartes indenfor 1 minut  & Som forventet & {\Huge \checkmark}\\ \midrule
	& \textit{Performance} & & & & \\ \midrule
	& Der skal vises en EKG-graf i interfacet, hvor spænding vises op ad y-aksen (-1V til 1V) og tiden på x-aksen & Gennemfør en måling & At spændingen for EKG-signalet er op ad y-aksen, samt tiden hen ad x-aksen & Spændingen er op ad y aksen og tiden i sekunder hen ad x-aksen. Dog er intervallet ikke -1V til 1V, se fejlrapport i bilag & {\Huge \ding{55}}\\ \midrule
	& Det skal være muligt at kunne scrolle igennem målingerne hen ad x-aksen & Der gennemføres en måling hvorefter der scrolles hen ad x-aksen & At der ved scrolling kan ses forskellige dele af EKG-signalet hen ad x-aksen & & {\Huge \checkmark}\\ \midrule
	& \textit{Supportability} & & & & \\ \midrule
	& Softwaren er opbygget af trelagsmodellen & Kig i koden efter data-lag, logik-lag og GUI-lag & At koden indeholder et data-lag, et logik-lag og et GUI-lag & Som forventet & {\Huge \checkmark}\\ \bottomrule
\caption{Accepttest af Ikke-funktionelle krav}
\end{longtabu}

