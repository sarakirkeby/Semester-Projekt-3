\chapter{Acceptest}\label{Accepttest}

\begin{longtabu} to \linewidth{@{}l l l X[j]@{}}
    Version &    Dato &    Ansvarlig &    Beskrivelse\\[-1ex]
    \midrule
  
\end{longtabu}

\section{Indledning}
I accepttesten testes de krav, der er opstillet i kravspecifikationen. Det fremgår af accepttesten om hvilke scenarier og krav, der er blevet opfyldt og implementeret i systemet og hvilke, der enten ikke er testbare eller ikke er blevet implementeret. \\
Først vil der være accepttest af use casene altså de funktionelle krav efterfulgt af accepttest af de ikke funktionelle krav. 

\section{Accepttest af Use Cases}


%Use case 1 acceptest
\subsection{Use Case 1}
\textbf{Kalibrer System}

\begin{longtabu} to \linewidth{@{} c X[j] X[j] X[j] l@{}}
    ~ &	Test &    Forventet resultat &		Faktiske observationer &    Godkendt\\[-1ex]
    \midrule
    ~ &\textit{Hovedscenarie} & ~ & ~ &
    \\ \midrule
	1. & Forbind systemet til en kendt trykkilde &    Systemet er forbundet til en kendt kilde &     &		%{\Huge \checkmark}
 \\   
 	2. & Aflæs spændingen målt på udgangen af det analoge filter &    At der forekommer en værdi, som efterfølgende kan bruges til lineærregression &     &		%{\Huge \checkmark}
 \\
 	3. & Skriv resultatet ned og gentag punkt 1 og 2. med to nye tryk &    Der er tre værdier skrevet ned &     &		%{\Huge \checkmark}
 \\
 	4. & Foretag lineærregression &   Afvigelseskoefficient er fundet &     &		%{\Huge \checkmark}
 \\
 	5. & Indtast afvigelses koefficient i tekstboks i "Log ind"\-vinduet &    Værdien er synlig i tekstboksen &     &		%{\Huge \checkmark}
\\
 	6. & Tryk på "Kalibrer"\-knappen &    Systemet er kalibreret &     &		%{\Huge \checkmark}
    
 \\ \bottomrule
\caption{Accepttest af Use Case 1.}\\
\label{AT_UC1}
\end{longtabu}

%Use case 2 acceptest 
\subsection{Use Case 2}
\textbf{Opstart system}

\begin{longtabu} to \linewidth{@{} c X[j] X[j] X[j] l@{}}
	& Test	& Forventet resultat		& Faktiske observationer		& Godkendt\\[-1ex] 
	\midrule
	&\textit{Hovedscenarie} & & & 
	\\ \midrule
	1. & Tryk på "Nulpunksjuster"\--knap & Panel vises med intruktioner om nulpunktsjustering & & %{\Huge \checkmark}
	\\
	2. & Tryk på "indlæs tryk"\--knap & Et tryk vises tekstboksen & & %{\Huge \checkmark}
	\\
	3. & Tryk på "Godkend"\--knap & Panelet lukkes & & %{\Huge \checkmark}
	\\
	4. & Indtast personale-ID i brugenavnsfeltet; "1234" og indtast personlig kode i kodeordsfeltet; "fido" & Loginoplysninger bliver udfyldt & & %{\Huge \checkmark}
	\\
	5. & Tryk på "Log ind"\--knappen & Log ind oplysninger er gyldige og stemmer overens med hinanden. "Blodtryks"\-vinduet vises  & & %{\Huge \checkmark}
	\\
	\\ \midrule
	& \textit{Undtagelser} & & &\\ \midrule
	5.a & Log ind oplysninger findes ikke i databasen. Besked vises med tekst, der informerer herom & Nye log ind oplysninger indtastes &  & %{\Huge \checkmark}
	\\ \bottomrule

\caption{Accepttest af Use Case 2.}\\
\label{AT_UC2}	
\end{longtabu}

%Use Case 3 acceptest

\subsection{Use Case 3}
\textbf{Mål blodtryk}

\begin{longtabu} to \linewidth{@{} c X[j] X[j] X[j] l@{}}
    ~ &	Test &    Forventet resultat &		Faktiske observationer &    Godkendt\\[-1ex]
    \midrule
    ~ &\textit{Hovedscenarie} & ~ & ~ &
    \\ \midrule
    1. & Tryk på "start"\--knappen i i blodtryksvinduet &    Graf og blodtryks værdier vises på brugergrænsefladen  &    &		%{\Huge (\checkmark)} 
 \\ \bottomrule
 
\caption{Accepttest af Use Case 3.}\\
\label{AT_UC3}
\end{longtabu}

%Use Case 4 acceptest

\subsection{Use Case 4}
\textbf{Filtrer signal}

\begin{longtabu} to \linewidth{@{} c X[j] X[j] X[j] l@{}}
    ~ &	Test &    Forventet resultat &		Faktiske observationer &    Godkendt\\[-1ex]
    \midrule
    ~ &\textit{Hovedscenarie} & ~ & ~ &
    \\ \midrule
    1. & Påsæt støjsignal med 50 mHz &    Signal vises i grafen &    &		%{\Huge \checkmark}
    \\
    2. & Tryk på "Til"\--knappen under filter&   Signalet udglattes &      &		%{\Huge \checkmark}
    \\
    3. & Tryk på ”Fra”-knappen under filter & Signal udglattes ikke &	& %{\Huge \checkmark}
	
 \\ \bottomrule
 
\caption{Accepttest af Use Case 4.}\\
\label{AT_UC4}
\end{longtabu}

%Use Case 5 acceptest

\subsection{Use Case 5}
\textbf{Alarmer bruger}

\begin{longtabu} to \linewidth{@{} c X[j] X[j] X[j] l@{}}
    ~ &	Test &    Forventet resultat &		Faktiske observationer &    Godkendt\\[-1ex]
    \midrule
    ~ &\textit{Hovedscenarie} & ~ & ~ &
    \\ \midrule
    1. & Indstil cursor til 100 for diastolisk øvre grænse &    Der står 100 i pågældende tekst felt &     &		%{\Huge \checkmark}
   	\\
   	2. & Indstil cursor til 80 for diastolisk nedre grænse &    Der står 100 i pågældende tekst felt &     &		%{\Huge \checkmark}
   	\\ 
   	3. & Indstil cursor til 100 for systolisk øvre grænse &    Der står 120 i pågældende tekst felt &     &		%{\Huge \checkmark}
   	\\
   	4. & Indstil cursor til 80 for systolisk øvre grænse &    Der står 80 i pågældende tekst felt &     &		%{\Huge \checkmark}
 	\\ 
	5. & Tryk på 'Start' knappen i blodtryksvindue &    Blodtryksmåling startes og Alarm går igang &     &		%{\Huge \checkmark}
 	\\ 
 	 \midrule
 	& \textit{Undtagelser} & & & \\
 	\midrule
 	5a & Tryk på "udskyd alarm"\-knappen & Alarmen udskydes med 3 minutter & &%{\Huge \checkmark}
 	\\
 	\bottomrule
\caption{Accepttest af Use Case 5.}\\
\label{AT_UC5}
\end{longtabu}

%Use Case 6 acceptest

\subsection{Use Case 6}
\textbf{Afslut system}

\begin{longtabu} to \linewidth{@{} c X[j] X[j] X[j] l@{}}
    ~ &	Test &    Forventet resultat &		Faktiske observationer &    Godkendt\\[-1ex]
    \midrule
    ~ &\textit{Hovedscenarie} & ~ & ~ &
    \\ \midrule
    1. & Tryk på "Afslut måling"\--knappen &   "Gemme"\--vinduet vises &     &		%{\Huge \checkmark}
   	\\
   	2. & Indtast CPR-nr "1111111118" &   CPR-nummeret synligt i pågældende tekst felt &     &		%{\Huge \checkmark}
   	\\
   	3. & Tryk på "Gem"\--knappen &   Bekræftigelse vises &     &		%{\Huge \checkmark}
   	\\ 
   	3. & Tryk på "Ok"\-knappen &  "Gemme"\--vindue og "Blodtryks"--vinduet lukkes. "Login"\--vinduet vises  &     &		%{\Huge \checkmark}
   	\\ \midrule
   	
	~ &\textit{Undtagelser} & ~ & ~ & 
	\\ \midrule
	1.a. & Tryk på "annuller"\--knap & "Gemme"\--vinduet lukkes og "Blodtryk"\-- vinduet vises &		&%{\Huge \checkmark}
	\\
	2.a. & Indtast CPR-nummeret "1234567890" & Beskeden "CPR ikke gyldigt." vises\ & 		& %{\Huge \checkmark}	
 \\ \bottomrule
 
\caption{Accepttest af Use Case 7.}\\
\label{AT_UC7}
\end{longtabu}


\section{Accepttest af ikke-funktionelle krav}

\begin{longtabu} to \linewidth{@{} c X[j] X[j] X[j] X[j] l@{}}
	& Ikke-funktionelt krav & Test/handling & Forventet resultat & Faktiske observationer & Godkendt
	\\[-1ex] \midrule
	&  \textit{Functionality} &  &  & & \\ \midrule
	1. & Brugeren skal kunne starte en ny måling indenfor 30 sekunder efter opstart & Start programmet, hvorefter der vha. stopur måles opstartstiden & At programmet er opstartet og ny måling er igang efter 30 sekunder &  & %{\Huge \checkmark}
	\\ \midrule
	2. &  Systemet skal kunne kalibrere blodtrykssignalet & Opstart programmet, til den automatiske kalibrering er fuldført & At systemet har kaliberet signalet & & \\ \midrule
	3. & Systemet skal kunne foretage en nulpunktsjustering & Tryk på "nulstil"\--knap & At blodtryksgrafen bliver nulpunktsjusteret & & \\ \midrule
	4. & Systemet skal kunne forstærke signalet med det indbyggede analoge antialiaserings filter med en båndbredde på 50 Hz & Start systemet & At signalet er forstærket & & \\ \midrule
	5. & Programmet skal kunne vise blodtrykket som funktion af tiden & Tryk på "Start måling"\--knap & At blodtrykket er vist som funktion af tiden på brugergrænsefladen & & \\ \midrule
	6. & Programmet skal kunne vise blodtrykssignalet kontiunert & Tryk på "Start måling"\--knap & At blodtrykssignalet er vist kontinuert på brugergrænsefladen & & \\ \midrule
	7. & Programmet skal programmeres i C\# & Start programmet & At koden er i C\# & & \\ \midrule
	8. & Programmet skal kunne lagre de målte data i en database & Tryk på "Gem"\--knap & At målingen er gemt i database & & \\ \midrule
	9. & Programmet skal kunne filtrere blodtrykket via et digitalt filter & Tryk på "Filtrer signal" til på radiobutton & At det viste blodtrykssignal er filtreret & & \\ \midrule
	10. & I programmet skal det digitale filter kunne slås til og fra på en radiobutton & Tryk "Filtrer signal" til og fra på radiobutton & & \\ \midrule
	11. & Programmet bør kunne afbilde både systolisk og diastolisk blodtryk med tal & Tryk "Start måling" & At systolisk og diastolisk blodtryk er afbilledet med tal på brugergrænseflade & & \\ \midrule
	12. & Programmet bør kunne måle og afbillede puls & Tryk på "Start måling" & At pulsen er afbilledet på brugergrænseflade & & \\ \midrule
	13. & Programmet bør kunne give alarm, hvis det systoliske blodtryk overstiger 140 mmHg eller falder under 100 mmHg & Påsæt blodtryksignal fra Physionet der overskrider valgte grænser & At en alarm (windowslyd beeb )begynder & & \\ \midrule
	14. & Programmet bør kunne give alarm, hvis det diastoliske blodtryk overstiger 90 mmHg eller falder under 40 mmHg & Påsæt blodtryksignal fra Physionet der overskrider valgte grænser & At en alarm (windowslyden beeb begynder & & \\ \midrule
	15. & Programmet kan angive pulsslag med bib-lyde med varighed af 100 ms og en frekvens på 850 Hz & Tryk på "Start måling" & At pulsen høres som bib-lyde af 100 ms varighed og med en frekvens på 850 Hz & & \\ \midrule
	& \textit{Usability} &  &  & & \\ \midrule
	1. & Blodtrykstallene der udskrives på brugergrænseflade er røde & Tryk "Start måling"\--knap & At Blodtrykstallene er røde & &  \\ \midrule
	2. & Pulsmålingen skal udskrives på brugergrænsefladen med grønne tal & Tryk "Start måling"\--knap & At pulsen vises med grønne tal & &  \\ \midrule
	3. & Brugergrænseflade lever op til nedensående figur & Opstart program og log ind & At brugergrænseflade indeholder samtlige funktioner som på figuren & &  \\ \midrule
	& \textit{Reliability} & & & & \\ \midrule
	1. & Systemet skal kunne køre uden fejl i et år & Start system op og vent et år & At programmet efter et år kører fejlfrit & & \\ \midrule
	2. & Systemet skal have en "mean time to restore" på højest 24 timer & Start systemet og herefter genstart, hvor der tages tid med et stopur & At programmet er klar igen inden for 24 timer & & \\ \midrule
	& \textit{Performance} & & & & \\ \midrule
	1. & Systemet bør kunne gemme data på 5 sekunder +/- 10\% & Tryk på "Gem og afslut"\--knap og tag tid med stopur & At data er inden for 5 sekunder & & \\ \midrule
	& \textit{Supportability} & & & & \\ \midrule
	1. & Softwaren er opbygget af trelagsmodellen & Kig i koden efter data-lag, logik-lag og GUI-lag & At koden indeholder et data-lag, et logik-lag og et GUI-lag & & \\ \bottomrule
\caption{Accepttest af Ikke-funktionelle krav}
\end{longtabu}

