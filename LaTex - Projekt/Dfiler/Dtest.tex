\chapter{Test}\label{Test}
\setcounter{secnumdepth}{5}

\begin{longtabu} to \linewidth{@{}l l l X[j]@{}}
    Version &    Dato &    Ansvarlig &    Beskrivelse\\[-1ex]
    \midrule
\label{version_Systemark}
\end{longtabu}

\section{Indledning} 
Test afsnittet giver en beskrivelse af hvilke test der løbende er blevet foretaget på både hardware og software. Testene består af modul test, hvor enkelte delelementer af systemet uafhængigt af hinanden. Modul testen er til for at verificere de enkelte enheder virker efter hensigten.\\
En mere omfattende integrationstest, hvor de større dele af både hardware og software testes, afhængigt af hinanden. Ved integrationstesten ligger fokus på hvordan systemet fungerer, set fra udviklers synspunkt.\\ Begge test ligger til grund for den endelige accepttest, hvor testen ses fra kundens synspunkt.

\section{Modul test hardware}\label{ModulHard}

Til test af hardwaredelen blev der først udført modultest på henholdsvis forstærkeren og det analoge filter. Efterfølgende blev der udført integrationstest på forstærkeren og det analoge filter sat sammen med tryk transduceren. Slutteligt blev den samlede hardwaredel bestående af tryktransducer, forstærker og analogt filter testet på en vandsøjle.

\subsection{Modul test af forstærkeren}
I modultesten af forstærkeren blev Analog Discovery brugt som spændingsforsyning og oscilloskop på to forskellige målepunkter placeret ved henholdsvis indgangs signalet og udgangssignalet.\\
Da tryktransduceren forventes at have output spændinger i området 0 til 6,25 mV blev signalet fra transduceren simuleret ved en sinus på 1 Hz med et offset på 0. Den laveste amplitude for indgangs signalet var 1 mV. For hver måling blev amplituden for indgangs signalet sat op med 1 mV indtil den sidste måling hvor amplituden var nået op på 10 mV. Således blev der 10 forskellige målinger. Reelt set var det kun nødvendigt at teste op til 6 mV eller 7 mV, da det er herimellem, at man kan forvente et max tryk fra transduceren at ligge. Da der alligevel er taget målepunkter op til 10 mV skyldes det at der ved flere målinger kan laves en ”pænere” tendenslinje ved lineær regression.

\begin{figure}[H]
	\centering
	\includegraphics[width=0.5\textwidth]{Figurer/Hardware/ForstaerkerTest}
	\caption{Måleopstilling ved modul test af forstærkeren.}
	\label{fig:ForstaerkerTest}
\end{figure}

Ud af målingerne blev der foretaget lineærregression over de 10 målepunkter. Tendenslinjen der kom ud af den lineæreregression blev som følger:

\begin{center}
\begin{align}
 Y = 415,8\cdot x-0,09 \text{     hvor }R^{2} = 0,99
\end{align}
\end{center}

Her kan Y beskrives som værende forstærkningen ved en given frekvens. Konstanten 415,8 er den reelle forstærkning som er målt for forstærkeren. Skæringen med y-aksen burde være 0, men grundet måleusikkerheder er den blevet -0,09, hvilket også er acceptabelt. Den høje R$^2$ værdi indikerer, at der er en tydelig lineær sammenhænge mellem den påtrykte spænding og spændingen af output, dvs. forstærkningen er lineær.\\
Senere påtryktes samme målopstilling et DC-signal med amplituder startende fra 3mV op til 10 mV. Igen blev amplituden øget med 1 mV for hvert måling således, at der blev 7 målepunkter. Ved målinger foretaget på signaler med både 1 mV og 2 mV er offsettet i Analog Discovery så stort, at målingerne er umulige
Efter at have foretaget lineærregression på de 7 målepunkter så tendenslinjen ud som følger:

\begin{center}
\begin{align}
Y = 386,38\cdot x+0,5547\text{     hvor }R^{2} = 0,998
\end{align}
\end{center}

Forstærkningen givet ud fra de 7 måle punkter er på 386 hvilket er en del under den målte forstærkning når systemet påtrykkes et sinussignal i stedet for et DC-signal. Dette kan skyldes at der i målingen af DC-signalet er færre målepunkter, og at det dermed bliver en mere usikker måling, da selv en mindre måleusikkerhed har en større indflydelse på den målte forstærkning.

\subsection{Modul test af det analoge filter}
I modultesten af det analoge filter blev Analog Discovery brugt som spændingsforsyning, signal generator og som oscilloskop til at måle amplituderne for input signalet og output signalet for filteret.

\begin{figure}[H]
	\centering
	\includegraphics[width=0.5\textwidth]{Figurer/Hardware/FilterTest}
	\caption{Måleopstilling ved modultest af det analoge filter.}
	\label{fig:FilterTest}
\end{figure}

I modultesten af det analoge filter ændres frekvensen for det påtrykte sinussignal på filter indgangen. Der blev målt på i alt 21 målepunkter som lå i intervallet 1 Hz til 500 Hz, se bilag XX for de præcise angivelser af de udvalgte målefrekvenser. Output amplituden samt $\Delta$t mellem de to grafer blev aflæst på de to grafer. På baggrund af disse målinger blev dæmpningen af signalet afbildet. 

\begin{figure}[H]
	\centering
	\includegraphics[width=1\textwidth]{Figurer/Hardware/AnalogScreenFilterAmp}
	\caption{Aflæsning af amplitude størrelse for outputsignalet, C2, fra filteret når inputsignalet, C1, har en amplitude på 2,5 V}
	\label{fig:FilterAmplitude}
\end{figure}

Herefter blev tidsforskydningen aflæst således at fasedrejet kunne udregnes.

\begin{figure}[H]
	\centering
	\includegraphics[width=1\textwidth]{Figurer/Hardware/AnalogScreenFilter}
	\caption{Aflæsning af tidsforskydningen for outputsignalet, C2, fra filteret set i forhold til indgangssignalet C1}
	\label{fig:FilterTidsforskydning}
\end{figure}

Det gælder generelt for et 2. ordensfilter af standarttypen som der er arbejdet med igennem projektet, at fasen er 0$^{\circ}$ en dekade før filterets knækfrekvens og falder med -90$^{\circ}$/dekade frem til dekaden efter knækfrekvensen. Derfor skal filteret i praksis have et fasedrej på minus -90$^{\circ}$ ved knækfrekvensen 50 Hz. Som det ses ud graf!! er fasedrejet ved frekvensen 5 Hz -5,2$^{\circ}$, denne skulle reelt set have været 0$^{\circ}$. 
Ved knækfrekvensen som er sat til 50 Hz er fasedrejet -86,4$^{\circ}$, hvor fasedrejet skulle have været -90$^{\circ}$. Ved målingen for 53 Hz er fasedrejet -95,4$^{\circ}$. Dermed kan der argumenteres for, at filterets reelle knækfrekvens må ligge et sted imellem 50 Hz og 53 Hz. 
For målingen på 500 Hz er fasedrejet -180$^{\circ}$, hvilket stemmer overens med teorien for 2. ordensfilterets fasekarakteristik.

\section{Modul test software}

Til test af softwaren er der først og fremmest foretaget modul test af  kodeelementer, for at teste om de fungerer efter hensigten. Dette er gjort ved hjælp af debug funktionen i Visual Studio. Disse test er foretaget løbende i udviklingsprocessen, hver gang en metode er tilføjet eller ændret. De væsenligste dele er nedenfor beskrevet: \\[1ex]
Login funktionen er den første funktion, er testet ved sammenligne tilladte brugernavne og kodeord i databasen med indtastede. Dvs. at kun de brugernavne og kodeords par der findes i databasen, som giver adgang til systemet.\\[1ex]
Nulpunktsjusteringen er testet ved at påsætte systemet en DC spænding på 2 Volt, hvorefter det indlæste tryk aflæses og sammenlignes med de 2 Volt. \\
I forbindelse med test af nulpunktjusteringen, hvor der påsættes 2 Volts DC spænding, er kalibreringen også testet ved at indtaste kalibreringsfaktoren på 2. Herefter er tjekkes at signalet er forstærket 2 gange.\\[1ex]
Testene foretaget i forbindelse med indlæsning af blodtrykssignal og detektering af systole, diastole og puls er alle foretaget ved følgende:\\
Der indsættes et kendt blodtrykssignal, med kendt antal toppe og dermed kendt systole, diastole og puls. De kendte værdier sammenlignes med værdierne der vises grafisk, samt med værdierne der findes ved debugging af systemet.\\[1ex]
Alarmen er tjekket ved at påsætte et blodtrykssignal med kendte blodtryksværdier: systolisk > 160 eller < 100 og diastolisk > 100 eller under 40), hvorefter det er testet om alarmen lyder ved de fire tilfælde uafhængigt af hinanden.\\ Yderligere for alarmen er der justering af grænseværdierne testet, ved at ændre disse grænseværdier således ovenstående værdier ikke er kritiske, for herefter at tjekke om alarmen lyder.\\ Ved alarm, er det testet at lyden kan udskydes med 3 minutter, som skrevet i koden, ved at lade blodtrykssignalet overskride grænseværdierne i mere end 3 minutter.\\[1ex]
Gemmefunktionen er testet ved at sende et kendt blodtrykssignal igennem systemet, og tjekke om de data der gemmes i databasen stemmer overen med de kendte data. I samme forbindelse er CPR tjekker testet ved først at indskrive et gyldigt CPR nummer, som først tjekkes i CPR-tjekker metoden og dernæst i databasen. Derefter er samme procedure gentaget med et ikke gyldigt CPR-nummer.


\section{Integrationstest hardware}


\section{Integrationstest software}

\section{Integrationstest system}
