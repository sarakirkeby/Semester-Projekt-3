\chapter{Resumé}
Gennem dette projekt er der arbejdet med udarbejdelse af en blodtryksmåler til en situation på en operationsstue. Produktet er udviklet som en invasiv blodtryksmåler, som medfører en præcis og kontinuerligt blodtryksmåling, hvilket er en fordel i forhold til casen. \\
Til udviklling af systemet er benyttet både SySML og UML til system, software og hardware beskrivelse. Udviklingsmetoder som V-modellen til udvikling og test.\\ 
Produktet er udviklet som en prototype, bestående af en hardware del såvel som en software del.\\[0.5ex]
Hardwaredelen sørger for at få forstærket et blodtrykssignal op til en størrelse der kan arbejdes med i softwaren, samt at udglatte signalet, som skal illustreres grafisk på en tilhørende computerskærm.\\[1ex]
Funktionerne i softwaredelen sørger for den grafiske visning af et blodtrykssignal, en valgfri yderligere udglatning af signalet, samt detektion af puls, systolisk- og diastoliskt blodtryk. Der er desuden implementeret en funktion, der ved kritisk for højt eller lavt blodtryk alarmerer via både lyd og grafik. Det sundhedsfaglige personale kan justere alarm værdier, således de tilpasses det enkelte individ. Data fra blodtryksmålingen gemmes efterfølgende i en privat database.\\[0.5ex]
Projektet har opfyldt de overordnede krav, hvilket er dokumenteret i accepttesten. \\[2ex]
\textbf{Abstract}\\
This project deals with development of an invasive blood pressure monitor that meet the conditions of an operating theater.\\
The product performs a precise and continuous blood pressure measurement, which is an advantage in relation to the given conditions.\\
SysML and UML have been used in the development process for describing the overall system including software and hardware. The V-model has been used for development and testing.\\
The product has been implemented as a prototype consisting of a hardware part and a software part.\\
The hardware part smoothens and amplifies an electrical blood pressure signal to a gain that makes it suitable for the software to display it on a monitor.\\
The software part also provides a choice of further smoothing of the signal as well as detection of pulse, systole and diastole. An alarm, which informs the user in case of hypertension or hypotension graphically and sonically, has been implemented as well. The healthcare professionals are then able to adjust the thresholds in order to fit the individual patient. Afterwards, the measurement are stored in a private database.\\
Through tests it has been concluded that the system are capable of fulfilling the given requirements.
