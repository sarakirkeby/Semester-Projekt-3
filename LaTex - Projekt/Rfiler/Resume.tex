\chapter{Resumé}
Gennem dette projekt er der arbejdet med udarbejdelse af en blodtryksmåler til en situation på en operationsstue. Produktet er udviklet som en invasiv blodtryksmåler, som medfører en præcis og kontinuerligt blodtryksmåling, hvilket er en fordel i forhold til casen. \\
Til udviklling af systemet er benyttet både SySML og UML til system, software og hardware beskrivelse. Udviklingsmetoder som V-modellen til udvikling og test.\\ 
Produktet er udviklet som en prototype, bestående af en hardware del såvel som en software del.\\[0.5ex]
Hardwaredelen sørger for at få forstærket et blodtrykssignal op til en størrelse der kan arbejdes med i softwaren, samt at udglatte signalet, som skal illustreres grafisk på en tilhørende computerskærm.\\[1ex]
Funktionerne i softwaredelen sørger for den grafiske visning af et blodtrykssignal, en valgfri yderligere udglatning af signalet, samt detektion af puls, systolisk- og diastoliskt blodtryk. Der er desuden implementeret en funktion, der ved kritisk for højt eller lavt blodtryk alarmerer via både lyd og grafik. Det sundhedsfaglige personale kan justere alarm værdier, således de tilpasses det enkelte individ. Data fra blodtryksmålingen gemmes efterfølgende i en privat database.\\[0.5ex]
Projektet har opfyldt de overordnede krav, hvilket er dokumenteret i accepttesten. \\
\textbf{Abstract}\\
\\
