\chapter{Projektbeskrivelse}

\section{Projektgennemførelse}\label{Projektgennemfoerlse}
Projektet startede med, at der blev lavet en tidsplan, som var mulig at ændre undervejs, dog med faste deadlines, som skulle overholdes. De forskellige deadlines lagde op til, at der kunne arbejdes efter udviklingsmodeller, som er beskrevet nærmere i metodeafsnittet \ref{Metode}.\\ \\
Tidsplanen blev sidenhen ført mere detaljeret ind i projektstryringsværktøjet Scrum. Scrum blev benyttet til at holde overblikket over manglende opgaver, igangværende opgaver og afsluttede opgaver. \\
Gruppens seks medlemmer blev fra start delt op i to undergrupper, én med hovedfokus på hardware udvikling, og én med hovedfokus på software udvikling. Dog blev de basic delene til projektet, som kravspecifikation og case udvalgt samlet. Scrum er her også et godt værktøj til at bevare overblikket over de to gruppers individuelle opgaver.\\ \\
Fra start blev der aftalt et ugentlig møde, med vejleder og de to grupper som medvirkende parter. På denne måde blev alle parter holdt opdateres på udviklingsprocessen, især grupperne imellem, men også vejleder. Sidst i forløbet, under test af diverse dele af systemet, blev grupperne samlet og testene blev udarbejdet i fællesskab.\\ \\
Projektet er gennemført ved udarbejdelse af en samarbejdsaftale, herunder udvælgelse af en projektleder, som i tilfælde af uoverensstemmelse havde den afgørende stemme. 

\section{Metode} \label{Metode}
\subsection{Ase-modellen}
Den primære udviklingsmodel, der er benyttet i dette projekt, er ASE modellen. ASE modellen er en udviklingsmodel, der tager udgangspunkt i use cases. 
\begin{figure}[H]
	\centering
	\includegraphics[width=1\textwidth]{Figurer/Metode/ASEmodellen}
	\caption{Projektmodel illustreret med de faser som projektet gennemløber\protect\footnotemark}
	\label{ASEmodel}
\end{figure}
\footnotetext{Fra \textit{"Vejledning til udviklingsprocessen for projekt 2"}}


Modellen er opbygget sådan, at udviklerne benytter vandfaldsmodellen (se afsnit \ref{Vandfald}) til at fastlægge en opgaveformulering, kravspecifikation og systemarkitektur, for derefter at designe og implementere de enkelte moduler i iterationer. \\ Ud fra projektformuleringen specificeres kravspecifikationen som en række use cases. Use cases er et værktøj, der beskriver diverse aktørers interaktion med systemet. Ved at definere kravspecifikationen ud fra use cases, opnås et overblik over hvilke krav, der stilles til systemets endelige funktionalitet.\\ \\ Ud fra kravspecifikationen kan systemets accepttest udarbejdes. Efter kravspecifikationen er fastlagt, udarbejdes systemarkitekturen.\\ I systemarkitekturen uddeles systemets funktionalitet i moduler og deres grænseflader til resten af systemet bestemmes. Ud fra systemarkitekturen designes systemet ved at nedbryde det efter funktionalitet, som kan bindes til både hardware og software.
\subsection{Vandfald}\label{Vandfald}
Denne metode bygger pa at gøre en hel fase af arbejdet færdigt før den næste startes. Grafisk ser det ud som på figur \ref{Vandfaldsmodel}: \\

\begin{figure}[H]
	\centering
	\includegraphics[width=0.8\textwidth]{Figurer/Metode/Vandfald}
	\caption{Vandfaldsmodel}
	\label{Vandfaldsmodel}
\end{figure}
Projektet starter med en analyse, og så videre med de andre faser - design, implementering og test. Det er altså hele systemet, der arbejdes igennem i hver fase, og vandfaldet symboliserer, at der kun arbejdes i en retning, altså man kan ikke gå imod strømmen. Metoden benyttes, når opgaven er veldefineret og velkendt. \\
Projekt forløbet skal have en kort varighed, dvs. mindre end ca. 4 måneder, under velkendte forhold med hensyn til udviklings- og testmiljø, udviklingsmetodik, platforme etc. \cite{Projektledelse}

\subsection{V-model}
V-modellen er en model, hvor testen planlægges parallelt med udviklingen. Accepttesten planlægges detaljeret efter kravnalysen, altså kravspecifikationen, systemtest planlægges detaljeret efter system design, og integrationstesten planlægges detaljeret efter arkitektur design fasen. Unit/modul testen ligger dog uændret i forhold til den traditionelle strategi.
\begin{figure}[H]
	\centering
	\includegraphics[width=1\textwidth]{Figurer/Metode/Vmodel}
	\caption{V-model}
	\label{Vmodel}
\end{figure}
Testens praktiske udførelse er altså uændret i forhold til Ase-modellen og Vandfaldsmodellen, dvs. den ligger sidst i forløbet. Det betyder at testfaserne planlægges modsat den rækkefølge, de udføres i. Den største forskel for testerne er, at planlægningen baseres på de tidlige modeller af systemet, ikke på det færdige system. \\
 V-modellen udvides desuden med reviews og deadlines (se afsnit \ref{Projektgennemfoerlse}).

\section{Specifikation og analyse}



\section{Arkitektur}
I det følgende afsnit beskrives arkitekturen for systemet. Systemarkitekturen fungerer her som udviklingsramme for videre design og implementering. Her bliver systemets funktionalitet 
nedbrudt til overordnede moduler. \\
   Arkitekturen for projektet er foretaget i to dele - en hardware arkitektur og en software arkitektur. Arkitekturen beskriver opbygningen af systemet i form af diagrammer.  
   \subsection{software arkitektur}
   I software designet er der udarbejdet en domæne model, der giver et overblik over hele systemet.
 \begin{figure}[H]
	\centering
	\includegraphics[width=1\textwidth]{Figurer/ISE/Domaenemodel}
	\caption{Domænemodel}
	\label{domaenemodel}
\end{figure}
   I domæne-modellen er relationerne mellem aktørerne og systemets dele beskrevet med pile og vejledende tekster - dette skulle gerne give et større overblik over systemets funktionalitet. \\ 
   En domænemodel beskriver dog ikke, hvilken rækkefølge de forskellige handlinger sker i, og derfor er der udarbejdet sekvensdiagrammer for hver use case for systemet, som skal beskrive dette.\\
   Her ses sekvensdiaframmet for use casen "Mål blodtryk"
    \begin{figure}[H]
	\centering
	\includegraphics[width=1\textwidth]{Figurer/ISE/sdAppModelUC3}
	\caption{Sekvensdiagram UC3}
	\label{sekvensdiagram}
\end{figure}
Her ses det hvordan brugeren interagerer med brugergrænsefladen ved at starte blodtryksmålingen. Herefter bliver metoden til at starte blodtryksmålingen kaldt ned gennem logik- og datalag hvor efter målingen vises i en graf på brugergrænsefladen. Grafen bliver hele tiden opdateret med nye målinger.\\
Ud fra dette kan det ses hvordan brugerens interaktion med brugergrænsefladen sætter gang i metoder i software programmet. Skevensdiagrammet giver altså et overblik over hvordan softwaren er bygget op.\\
Sekvensdiagrammerne for de øvrige use cases kan ses i dokumentationen afsnit ....
  
 \subsection{SysML}
 I beskrivelsen af systemarkitekturen og det detaljerede design for det færdige produkt, er der 
anvendt SysML. SysML kommer originalt fra UML, men UML er hovedsagligt centreret omkring udvikling af software systemer. Da det udviklede system både består af hardware og software, er der valgt SysML til arkitekturen.\\
Valget af SysML grunder også i, at det giver en god formidling af systemet - dette giver udviklerne et større overblik. Samtidig er det også let for en udenforstående at sætte sig ind i systemets kunnen.\\

I dette projekt er der benyttet struktur- og adfærdsdiagrammer til at specificere og 
dokumentere systemet. Som strukturdiagram er der anvendt et blok definitions diagram (bdd) samt interne blok definitions diagramm (ibd).\\
Der er anvendt adfærdsdiagrammer i form af sekvensdiagrammer i dette projekt. Disse 
diagrammer er velegnet til sekventielt at beskrive den logiske funktionalitet i systemet.
Softwaren er opbygget ud fra sekvensdiagrammer beskrevet i arkitektur afsnittet. 



\subsection{Design}


\subsection{Implementering}
Herefter blev de to blokke bygget op. Gruppen valgte at bygge forstærkeren og filteret på hver sit fumlebræt, dels på grund af pladsmangel og dels på grund af større sammenhæng mellem arkitekturen, og det endelige produkt.\\

    \begin{figure}[H]
	\centering
	\includegraphics[width=1\textwidth]{Figurer/Hardware/samletopstilling}
	\caption{Opbygning af forstærker og filter}
	\label{samletopbygning}
\end{figure}

Grundet mangel på præcise modstande, bedømte gruppen at det var bedst at bygge modstandende i forholdsvist filteret og forstærkeren, op i to, så gruppen kunne komme så tæt på den ønskede modstandsværdi som muligt. \\

\subsubsection{Implementering af forstærkeren}
Den samlede stykliste for forstærkeren blev vist som på tabel \ref{forsttabel}.

\begin{center}
\begin{tabular}{|c|c|c|}
\hline 
Komponent & Antal & Type \\ 
\hline 
Modstand & 1 & 120\\
\hline
Modstand & 1 & 4,8\\
\hline
Kondensator & 2 & 100 nF\\
\hline
Instrumentationsforstærker & 1 & INA114 \\ 
\hline
\end{tabular} 
\label{forsttabel}
\end{center}

For forstærkeren gælder det at den beregnede $R_{gain}$ er 125,31\Ohm som det kan ses ud af komponentlisten består RGain i praksis af to modstande på henholdsvis 4,8Ω og 120Ω som er sat i serie. $R_{gain}$ modstanden er i praksis 124,8\Ohm. I praksis er $R_{gain}$ 0,51\Ohm mindre end den i teorien skulle have været.

\subsubsection{Implementering af filteret}
Den samlede stykliste for filteret blev som vist på tabel \ref{filtertabel}.

\begin{center}
\begin{tabular}{|c|c|c|}
\hline 
Komponent & Antal & Type \\ 
\hline 
Modstand & 2 & 6.2 k \\ 
\hline 
Modstand & 2 & 470\\
\hline
Kondensator & 1 & 680 nF\\
\hline
Kondensator & 1 & 330 nF\\
\hline
Operationsforstærker & 1 & OP27G \\ 
\hline
\end{tabular} 
\label{filtertabel}
\end{center}

Det analoge filter består blandt andet af en 330 nF kondensator, $C_{1}$, som i teorien er beregnet til at skulle have været 333,2 nF. I praksis er kondensatoren $C_{1}$ 3,2 nF mindre end den i teorien er beregnet til at skulle have været. Filteret består desuden af to modstande $R_{1}$ og $R_{2}$ som er identiske. I det realiserede analoge filter består hver modstand af to modstande på henholdsvis 6200\Ohm og 470\Ohm som er sat i serie. Dermed er både $R_{1}$ og $R_{2}$ 6670\Ohm i praksis. I teorien var $R_{1}$ og $R_{2}$ udregnet til at skulle være 6687\Ohm. I praksis er der derfor 17\Ohm mindre end teorien foreskriver.\

Generelt er der valgt at se bort fra de afvigelser, der er for komponentværdierne i praksis sammenlignet med de i teorien beregnet. Det er valgt da afvigelserne er relativt små i forhold til det pågældende komponent. For modstandende er der desuden 1 procents usikkerhed, hvilket betyder man alligevel ikke kan være helt sikker på komponentværdien.\

På baggrund af de i praksis anvendte komponenter er den reelle knækfrekvens for det analoge filter beregnet. Til det formål er formlen anvendt.\

CUTOFFFREKVENS!?

\subsection{Test}


\section{Resultater og diskussion}


\section{Opnåede erfaringer}


\section{Fremtidigt arbejde}
