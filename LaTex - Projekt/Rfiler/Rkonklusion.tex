\chapter{Konklusion}
I forgående projekt, er der blevet udviklet en blodtrykssystem prototype, bestående af både software og hardware, som er i stand til, at måle og vise intravenøst blodtryk. \\
\\
Gruppen startede med at fremlægge forholdsvist realistiske krav til prototypen. Disse krav blev dog ændret i løbet af udviklingsfasen, fordi nogen af kravene, ikke passede sammen. Dette gjorde sig især gældende i hardware kravene, som omhandlede forstærkning og forsyningsspænding. Også i software delen, var der dog behov for at sortere enkelte mindre designrelaterede krav fra.  \\
\\
Der er forsøgt fokuseret på, hvordan et sådan system, ville skulle fungere ude i virkeligheden. Dette kan ses på de krav, som gruppen selv har sat, ud over de fastlagte IHA krav. Efter kravene blev modificeret, er det dog lykkedes gruppen at leve op til alle de krav, som der blev sat til projektet. \\

\\Ideelle tanker kontra realitet??
\\
Da accepttesten blev gennemført, var der ingen behov for at skrive nogen fejlrapporter, da alle krav kunne eftervises og testes i programmet. Enkelte tekniske krav var dog nødt til, at blive eftervist i større omfang, i separate test afsnit. De større krav omkring at måle, filtrere og afbillede et blodstryks signal blev opfyldt til fulde. Alle mindre krav er desuden også implementeret med succes. \\
\\
Selve udviklingen af systemet, har været meget præget af, at gruppen har været delt i to. Der har været svag kommunikation imellem de to grupper, men på trods af dette er det lykkedes gruppen at få udviklet to dele, som kunne arbejde sammen. \\
\\
Generelt set, mener gruppen at projektet er gået godt, og at det system, som er blevet udviklet, er klar til at blive udviklet yderligere, skulle det være en mulighed. \\
\\
