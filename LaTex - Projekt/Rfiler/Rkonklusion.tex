\chapter{Konklusion}
I dette projekt er der blevet udviklet en blodtrykssystemsprototype, bestående af både software og hardware, som er i stand til at måle og vise invasivt blodtryk. \\[2ex]
Som start blev der udarbejdet krav til prototypen, som dog blev ændret i løbet af udviklingsfasen, da ikke alle krav passede sammen. Dette gjorde sig især gældende for hardware-kravene, hvor der ikke var taget højde for den maksimale ratio mellem forstærkning og forsyningsspænding. I software-delen var der behov for at sortere enkelte mindre designrelaterede krav fra.\\[2ex]
Det primære krav, at udvikle et blodtryksmålingssystem, er lykkedes. Der er forsøgt fokuseret på, hvordan et sådant system ville fungere i virkeligheden.\\ Dette kan ses på de krav, der er sat, som supplerer de fastlagte IHA-krav. Det er lykkedes gruppen at leve op til alle de krav, der blev sat til systemet, hvilket også kommer til syne i accepttesten. \\[2ex]
Da accepttesten blev gennemført, var der ingen behov for at skrive nogen fejlrapporter, da relevante krav kunne eftervises og testes i programmet. Enkelte tekniske krav var dog nødt til at blive eftervist i større omfang, i separate test afsnit. Alle mindre krav er desuden også implementeret med succes.\\[2ex]
Selve udviklingen af systemet, har været meget præget af, at gruppen har været delt i to. Der har i begyndelsen været begrænset kommunikation imellem software-gruppen og hardware-gruppen, men på trods af dette er det lykkedes at få udviklet to dele, som kunne arbejde sammen. \\[2ex]
Generelt set er projektet gået godt. Et system bestående af en forstærker, et filter og et C\#-program, som kan vise blodtryk kontinuerligt, foretage en digital filtrering og nulpunktsjustering, samt alarmere bruger, er blevet udviklet. Der er bred enighed om, at det system, som er blevet udviklet, er klar til videreudvikling.

