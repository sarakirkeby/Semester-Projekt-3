\chapter{Konklusion}
I dette projekt er der blevet udviklet en software prototype, som kan afbilde, analysere og gemme EKG-signaler. \\\\
Gruppen startede ud med høje ambitioner omkring prototypen, men opdagede hurtigt, at tankerne omkring, hvordan prototypen ideelt set skulle være og hvad der reelt kunne udvikles ikke stemte overens. En af de ting, som blev sorteret fra, som en del af de for høje ambitioner, var en funktion, der hentede gamle målinger frem og viste dem i interfacet. En anden ting, som også blev sorteret fra, var mulighed for at vælge tidsinterval for, hvor længe målingen skulle køre.\\ \\
På trods af de høje forventninger har gruppen formået at opfylde alle overordnede krav, som blev sat i starten af projektet. Dette kan ses på den gennemførte accepttest, som kun havde en enkelt designrelateret afvigelse. Kravene om at afbilde og analysere et EKG-signal er blevet opfyldt, og det er også lykkedes gruppen at implementere en funktion, som kan gemme i en privat database. \\ \\
Gruppen har oplevet at, det har været en udfordring at udvikle en analyse, som kunne dække over varierende signaler. Det blev klart, at et EKG-signal diagnosticeret med atrieflimren kan variere i udseende, og det blev, derfor svært for gruppen at udvikle en passende analyse. Analysen blev derfor udarbejdet ud fra et specifikt signal, dog passer denne analyse også på et sundt EKG-signal, således at patienten både kan erklæres syg og rask. \\ \\
Kravet fra Sundhedsstyrelsen, om en offentlig database, blev præsenteret sent i arbejdsprocessen, men det lykkedes gruppen at implementere den offentlige database uden store vanskeligheder. \\
Udviklingsprocessen har været præget af, at gruppen har tænkt meget over, hvordan dette system ville kunne fungere ude i virkeligheden. Dette kan ses på nogle af de krav, som gruppen selv har sat til projektet. Dette drejer sig bl.a. om login-funktionen, der giver mulighed for at identificere og validere brugeren af systemet. Derudover havde gruppen store krav til en række design relaterede funktioner, som bl.a. graf layout såsom gridlines og scroller.\\ \\
Projektet er overordnet set gået godt, og gruppen er tilfreds med slutproduktet, som gruppen mener er klar til videre udvikling.

