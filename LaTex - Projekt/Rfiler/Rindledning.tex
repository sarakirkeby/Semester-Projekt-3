\subsubsection{Versionhistorik}

\begin{longtabu} to \linewidth{@{}l l l X[j]@{}}
    Version &    Dato &    Ansvarlig &    Beskrivelse\\[-1ex]
    \midrule
   
    	
\label{version_Systemark}
\end{longtabu}

\chapter{Indledning}

En blodtryksmåling er i danmark en af de hyppigste udførte undersøgelser rundt omkring på hospitaler, i ambulancer, plejehjem, bloddoner centre osv. 
Den hyppige undersøgelse ratio skyldes bl.a. at forhøjet blodtryk er stort problem i Danmark, især hos den ældre del af befolkningen. Incidensen i Danmark er omkring 1 million, hvor mange af disse patienter har så let en blodtryksafvigelse, at man ikke giver medicinal behandling til at starte med. Mange i Danmarks befolkning går rundt med en blodtryksafvigelse uden at vide det, da de tilhørende symptomer ofte er nogen, der kan associeres med en anden årsag. Incidensen for blodtrykssygdomme er stigende med alderen og når 40\% ved de 60-69-årige.\cite{Statistik} \\
\\
En blodtryksmåling kan foretages både invasivt, intra-artielt, og noninvasivt, ved manchet. Fælles for begge metoder er at blodtryksapparatet leverer et spændingssignal, som ændres afhængigt trykket i årerne, hvorefter der monitoreres blodtryksværdier klassificeret ved systoliske og diastoliske værdier. Det samlede signal, er det, der bliver afbilledet på den typiske blodtryksmonitor.\\
Den noninvasive blodtryksmåling er den mest udbredte, og giver er her og nu billede af blodtrykket. Den invasive måling benyttes mere ved kliniske undersøgelser og operationer, hvor det er en fordel at få vist blodtrykket kontinuerligt på en graf\\
 Blodtryk anses for at være normalt, hvis det befinder sig under 140 mmHg systolisk, og under 90 mmHg diastolisk. \\
 \\
I dette projekt er formålet, på baggrund af viden om kredsløbet og blodtryksmålinger, at udvikle et system, som kan måle og detektere akutte blodtryks afvigelser på et måleobjekt.\\ 
Til dette er der udviklet et hardware, som skal forstærke og filtrere et blodtrykssignal, som efterfølgende skal vises kontinuerligt og analyseres i en udviklet software. Projektets scenarie er en blodtryksmåler til en operationsstue. Her er det nødvendigt at overvåge blodtrykket løbende. I tilfælde af pludselige blodtryksændringer, kan det betyde blødninger, shock og lignende. Systemet er derfor udviklet til at kunne give besked om pludselige ændringer i blodtrykket, samt at kunne gemme dokumentation herom. \\[1ex]
Rapporten er opsat efter normer for naturvidenskabelige rapporter, bestående af et baggrundsafsnit, med forklarende teori der er brugt til at drage konklusioner i rapporten, et kravsspecifikationsafsnit, og derefter et afsnit som informerer om projektudførelsen. \\[1ex]


  
