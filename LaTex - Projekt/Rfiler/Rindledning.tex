\subsubsection{Versionhistorik}

\begin{longtabu} to \linewidth{@{}l l l X[j]@{}}
    Version &    Dato &    Ansvarlig &    Beskrivelse\\[-1ex]
    \midrule
   
    	
\label{version_Systemark}
\end{longtabu}

\chapter{Indledning}

Forhøjet blodtryk er stort problem i Danmark, især hos den ældre del af befolkningen. Incedensen i Danmark er cirka 1 million, hvor mange af disse patienter har så let en blodtryksafvigelse, at man ikke giver medicinal behandling til at starte med. Mange i Danmarks befolkning går rundt med en blodtryksafvigelse uden at vide det, da de tilhørende symptomer ofte er nogen, der kan associeres med en anden årsag. Incedensen for blodtrykssygdomme er stigende med alderen og når 40% ved de 60-69-årige. \\
Blodtryksafvigelser bliver målt ved hjælp af et blodtryksapparat. Blodtryksapparatet kan være intra-arteriel eller ved hjælp af en manchet. Blodtryksapparatet leverer et spændingssignal, som ændres afhængigt trykket i årerne. Signalet er målt i mmHg, og klassificeres i det systoliske blodtryk og det diastoliske blodtryk. Det samlede signal, er det, der bliver afbilledet på den typiske blodtryksmonitor. Blodtryk anses for at være normalt, hvis det befinder sig under 140 mmHg systolisk, og under 90 mmHg diastolisk. \\
I dette projekt er formålet at udvikle et system, som kan detektere akutte blodtryks afvigelser på en operationsstue. Til dette er der blevet udviklet et stykke hardware, som skulle, forholdsvist, forstærke og filtrere et blodtrykssignal, hvorefter dette skulle vises og analyseres i et stykke software, som også skulle udvikles. I en operationsstue, er det især kritisk, hvis patientens blodtryk pludseligt falder, da dette kan betyde blødninger, chok eller andet. Systemet er derfor udviklet til at kunne give besked om pludselige ændringer i blodtrykket, samt at kunne gemme dokumentation herom. \\
Rapporten er opsat efter normer for naturvidenskabelige rapporter. Dette vil sige, at den består af et baggrundsafsnit, som forklarer den teori der er brugt til at drage konklusioner i rapporten, herefter er der givet et kravsspecifikationsafsnit, og derefter et afsnit som informerer om projektudførelsen. \\


  
