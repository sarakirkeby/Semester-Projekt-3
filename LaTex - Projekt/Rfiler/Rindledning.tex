\subsubsection{Versionhistorik}

\begin{longtabu} to \linewidth{@{}l l l X[j]@{}}
    Version &    Dato &    Ansvarlig &    Beskrivelse\\[-1ex]
    \midrule
   
    	
\label{version_Systemark}
\end{longtabu}

\chapter{Indledning}

En blodtryksmåling er i Danmark en af de hyppigst udførte undersøgelser på hospitaler, praktiserende læger, i ambulancer, plejehjem og bloddonorcentre. 
Den hyppige undersøgelsesratio skyldes bl.a., at forhøjet blodtryk er et stort problem i Danmark, især hos den ældre del af befolkningen. Incidensen i Danmark er omkring 1 million, hvor mange af disse patienter har en så minimal blodtryksafvigelse, at man ikke giver medicinal behandling til at starte med. Mange i Danmark går rundt med en blodtryksafvigelse uden at vide det, da de tilhørende symptomer ofte er nogen, der kan associeres med en anden årsag. Incidensen for blodtrykssygdomme er stigende med alderen og når op på 40\% ved de 60-69-årige.\cite{Statistik} \\
\\
En blodtryksmåling kan foretages både invasivt med nål og non-invasivt med manchet. Fælles for begge metoder er, at blodtryksapparatet leverer et spændingssignal, som ændres afhængigt af trykket i årerne, hvorefter der monitoreres blodtryksværdier klassificeret ved systoliske og diastoliske værdier. Det samlede signal er det, der bliver afbildet på den typiske blodtryksmonitor.\\
Den non-invasive blodtryksmåling er den mest udbredte og giver et her-og-nu-billede af blodtrykket. Den invasive måling benyttes mere ved kliniske undersøgelser og operationer, hvor det er en fordel at få vist blodtrykket kontinuerligt på en graf.\\
 Blodtryk anses for at være normalt, hvis det befinder sig mellem 100-140mmHg systolisk, og mellem 60-90mmHg diastolisk.\cite{normalt blodtryk} \\
 \\
I dette projekt er formålet, på baggrund af viden om kredsløbet og blodtryksmålinger, at udvikle et system, som kan måle og detektere akutte blodtryksafvigelser på et måleobjekt.\\ 
Til dette er der udviklet hardware, som skal forstærke og filtrere et blodtrykssignal, som efterfølgende skal vises kontinuerligt og analyseres i et udviklet program. Projektets scenarie er en blodtryksmåler til en operationsstue. Her er det nødvendigt at overvåge blodtrykket løbende. Blodtrykket kan få pludselige ændringer, hvilket bl.a. kan skyldes blødninger og shock. Systemet er derfor udviklet til at kunne give besked om disse ændringer i blodtrykket, samt at kunne gemme dokumentation herom. \\[1ex]
Rapporten er opstillet efter normer for naturvidenskabelige rapporter, bestående af et baggrundsafsnit med forklarende teori, der er brugt til at drage konklusioner i rapporten, et kravsspecifikationsafsnit, og derefter et afsnit som informerer om projektudførelsen. \\[1ex]


  
