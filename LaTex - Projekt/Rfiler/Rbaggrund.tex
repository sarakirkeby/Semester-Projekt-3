\chapter{Baggrund}
\section{Hjertet \& Kredsløb}
Hjertet, \textit{cor}, er en muskel, hvis opgave er at pumpe blodet rundt i kroppen. Hjertet er hult, og består af fire kamre. To atrier, og to ventrikler. Det er ventriklerne som fungerer som selve pumpen, hvor atrierne primært er et bassin for blod. Kamrene er afskilt af anulus fibrosus, bestående af fire ringe, som indeholder hjerteklapperne. Der er fire hjerteklapper, to AV-klapperne, aortaklappen og pulmonalklappen. Hjertets anatomi kan ses på figur \ref{Hjeret}:\\

\begin{figure}[H]
	\centering
	\includegraphics[width=0.8\textwidth]{Figurer/Fysio/Hjertet}
	\caption{Hjerte med forklarende pile \protect\cite{Hjertet}}
	\label{Hjeret} 
\end{figure}

På figur \ref{cyklus} kan faserne i hjertets cyklus ses. Disse inddeles i to hovedfaser, den første kaldt diastolen, og den anden systolen. I diastolen er ventriklerne afslappede, og trykket er her lavere, end det tryk der er i arterierne. Derved åbnes AV-klapperne og ventriklerne fyldes med blod. Herefter går hjertet i systole, hvor ventriklerne kontraherer. Trykket i ventriklerne stiger, hvilket lukker AV-klapperne og åbner aorta- og pulmonalklappen. Blodet strømmer herefter ud i årerne, indtil ventriklerne bliver afslappede igen, og hjertets cyklus startes forfra. \cite{cyklus}

\begin{figure}[H]
	\centering
	\includegraphics[width=1\textwidth]{Figurer/Fysio/Cyklus}
	\caption{De forskellige faser i hjertets cyklus \protect\cite{cyklus}}
	\label{cyklus}
\end{figure}

\section{Hæmodynamik}
Når blodet skal fra hjertet og rundt i kroppen, taler man om et blodflow. Blodets flow opfører sig som shear thinning fluid, hvilket gør sig gældende ved ikke-newtonske væsker med formindsket viskositet. At blodet hører under denne kategori, skyldes at erytrocytterne  organiseres ved et øget flow. \\
Når hjertet pumper, opstår der et tryk i blodkarrene. Blodtrykket er produktet af hjertets pumpearbejde og modstanden mod blodstrømmen i blodkredsløbet. Trykket er højest i arterierne og lavest i venerne.\cite{haemo}\\
Blodtrykket deles op i et systolisk tryk og et diastolisk tryk. Det systoliske tryk er det tryk, der opstår under hjertets sammentrækning, altså hjertets uddrivningsfase. Det diastoliske tryk opstår i hjertets afslapningsfase. I disse faser er det arterielle blodflow ikke steady, men derimod pulsatilt. 
\\Dog falder trykket ikke til 0 i diastolen pga. pulsårevæggenes elasticitet. Forholdet mellem tryk og volumen er illustreret i figur \ref{tryk og volumen}
\begin{figure}[H]
	\centering
	\includegraphics[width=0.8\textwidth]{Figurer/Fysio/TrykOgVolumen}
	\caption{Forhold mellem tryk og volumen\protect\cite{tryk}}
	\label{tryk og volumen}
\end{figure}
Figur \ref{tryk og volumen} viser yderligere, hvordan hjerteklappens lukning fungerer, når et trykfald herover ændrer retning. I det systemiske kredsløb er første kar aorta, som grundet sin elasticitet vil få størstedelen af blodmængden, som er pumpet ud af venstre ventrikel, til at blive dæmmet op i aorta. Dette medfører, at der oplagres en elastisk potentiel energi i aortavæggen. Denne energi udgør et tryk, der har indflydelse på og bidrager til et blodflow i diastolen efter aortaklappens lukning og hjertets uddrivningsfase. 

\section{Hypertension}
Hypertension defineres ud fra vedtagne blodtryksgrænser. De nuværende blodtryksgrænser ligger på et systolisk tryk over 140mmHg og/eller et diastolisk tryk på over 90mmHg. Disse grænserværdier gælder uanset patientens alder. Grænseværdierne er dog kun et udgangspunkt, for der kan godt opstå hypertension hos en person med et i forvejen for lavt blodtryk og i dette tilfælde vil grænseværdierne ikke nå op på værdien for definitionen af hypertension.\\
Hypertension medfører betydelig øget risiko for kardiovaskulære sygdomme, som oftest er apopleksi og iskæmisk hjertesygdom. Herudover kan hypertension medføre påvirkning af nyrerne.\cite{Hypertension}

\section{Hypotension}
Hypotension defineres som et vedvarende systolisk tryk under 100mmHg i hvile. \\
Under operationer og traumer er hypotension en mere alvorlig ting og defineres ofte som shock.\\
Shock er defineret ved en patofysiologisk tilstand karakteriseret ved, at blodcirkulationen er utilstrækkelig til at imødekomme kroppens metaboliske behov. Blodtryksgrænsen for shock angives forsimplet ofte at være systolisk blodtryk på under 90mmHg eller et fald i systolisk tryk på 40mmHg.
\cite{Hypotension}

\section{Blodtryksmåling}
For at kunne detektere et blodtryk som beskrevet i ovenstående, er det nødvendigt at foretage en blodtryksmåling. \\
Der findes mange former for blodtryksmålinger. Man skelner mellem non-invasive og invasive målinger. De non-invasive målinger kan være målemetoder som den klassiske blodtryksmåling med manchet, stetoskop og kviksølvsmanometer. Den invasive metode indebærer en indsættelse af instrument i kroppen og benyttes blandt andet på operationsstuer. Et invasivt blodtryksmålingsapparat kan deles op i to generelle metoder. Den meste brugte kliniske metode er at koble det vaskulære tryk til et eksternt sensorelement via et væskefyldt kateter. Den anden metode er en metode, hvor vandkoblingen bliver elimineret ved at inkorporere sensoren i spidsen af kateteret i det vaskulære system.\\
\\
\begin{figure}[H]
	\centering
	\includegraphics[width=0.6\textwidth]{Figurer/Fysio/tryk}
	\caption{Opstilling af invasiv blodtryksmåling \protect\cite{intramaaling} }
	\label{intramaaling}
\end{figure}

Som set på figur \ref{intramaaling}, placeres en nål invasivt på en patient. Nålen er forbundet til en trykpose med et natriumklorid-fyldt kateter, påsat en transducer. Posen har en udtømningsmekanisme, der sørger for, at der ikke er bobler i kateteret. Det interne tryk i posen bliver reguleret, til at ligge over patientens systoliske blodtryk. \\
Når trykket i posen reguleres, kan dette observeres på den tilkoblede monitor. Når trykreguleringen stoppes, fortsætter posen med at dryppe natriumklorid i kateteret, da det stopper blodet fra at fylde kateteret. Trykket fra patientens arterie kan nu aflæses, da trykket i kateteret stemmer overens med det tryk, der kan findes i arterien. \cite{intramaaling} 

\begin{figure}[H]
	\centering
	\includegraphics[width=0.6\textwidth]{Figurer/Fysio/trykmaaling}
	\caption{Grafisk afbildning af blodtryk \protect\cite{intramaalinggraf}}
	\label{grafiskmaaling}
\end{figure}

Et eksempel på en måling kan ses på figur \ref{grafiskmaaling}. Her ses det, at målingen består af en masse bølger. Disse bølger repræsenterer det samlede blodtryk, og hver bølge repræsenterer et pulslag. Toppen af bølgen repræsenterer det systoliske blodtryk, og minimum repræsenterer det diastoliske.  

\section{Sensorer}
En sensor er en transducer, der transformerer en fysisk målestørrelse til elektrisk energi. Til måling af fysiologiske størrelser som blodtryk, bruges sensorer som omformer tryk til elektrisk energi. Et eksempel på sådanne sensorer er en strain gauge som er en resistiv transducer. Strain gauges klassificeres enten som bundne eller ubundne, hvor den ubundne giver en temperaturkompensation, mens den bundne kan have udsvinging grundet temperaturen.\\
Den ubundne strain gauge består af fire sæt af strækfølsomme ledninger, der er forbundet, så de danner en Wheatstone bro, se figur \ref{StrainGauge}. Disse ledninger er monteret under tryk mellem rammen og det bevægelige armatur, således at den maksimale belastning, som strain gaugen kan holde til, er større end den forventede udefrakommende komprimerende belastning. Dette er nødvendigt for ikke at skade ledningerne. Disse typer af sensorer kan blive brugt til at konvertere blodtryk til membranbevægelse, videre til modstandsændring og til sidst et elektrisk signal. Brosammenkoblingen giver en temperaturkompensation, og den giver et fire gange så stort output fordi alle fire arme indeholder aktive gauges.\\

\begin{figure}[H]
	\centering
	\includegraphics[width=0.6\textwidth]{Figurer/Hardware/straingauge}
	\caption{}
	\label{StrainGauge}
\end{figure}








