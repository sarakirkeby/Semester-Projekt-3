\chapter{Baggrund}

\section{Hjertet}
Hjertet, \textit{cor}, er en hul muskel, der har til opgave at pumpe blodet rundt til hele kroppen. Hjertet består af i alt fire kamre, som det kan ses på figur 3.1 nedenfor. To forkamre, atrier, og to hjertekamre, ventrikler. Atrierne fungere primært som reservoir for blod, mens ventriklerne fungerer som den effektive pumpe.\\

\begin{figure}[htb]
	\centering
	\includegraphics[width=1\textwidth]{Figurer/Snip20150410_31}
	\caption{Hjerte med forklarende pile \protect\footnotemark} 
\end{figure}
\footnotetext{http://www.hjertelunge.dk/hjertesygdomme/hjerte\_og\_kredsloeb/hjertet/}

Hjertekamrene og forkamrene er adskilt fra hinanden af anulus fibrosus, som er en plade af bindevæv. Anulus fibrosus består af fire bindevævsringe, der er forbundet med hinanden. To af disse udgør åbningerne mellem atrierne og ventriklerne. De to sidste danner åbningerne mellem højre hjertekammer og lungepulsåren og venstre ventrikel og hovedpulsåren. Ved alle bindevævsringene er der klapper, der fungere som ventiler.\\ 
AV-klapperne sidder mellem atrierne og ventriklerne. Klappen mellem højre atrium og ventrikel kaldes tricuspidalklap, mens klappen mellem venstre atrium og ventrikel kaldes mitralklap, se figur 3.1. Aortaklappen er placeret ved afgangen af hovedpulsåren og pulmonalklappen ved afgangen af lungepulsåren. Klapperne fungere således, at blodet kun kan løbe én vej gennem dem. Åbningen samt lukningen af disse er en passiv proces, som bestemmes af forskelle i væsketrykket på de to sider af klapperne.\\ 

\begin{figure}[htb]
	\centering
	\includegraphics[width=1\textwidth]{Figurer/Snip20150412_7}
	\caption{De forskellige faser i hjertets cyklus \protect\footnotemark}
\end{figure}
\footnotetext{Billede fra "Menneskets anatomi og fysiologi" s. 273 figur 9.6}

Hjertets cyklus, som er illustreret ved figur 3.2, inddeles i to hovedfaser. Den første kaldes diastolen. I diastolen er ventriklerne afslappede og fyldes med blod. Det vil sige, at trykket i ventriklerne bliver lavere end trykket i atrierne, således at AV-klapperne åbnes, og blodet begynder at strømme ind i ventriklerne. Under hele diastolen er aortaklappen lukket. Den anden fase kaldes systolen. I systolen kontraherer ventriklerne sig. Trykket i ventriklerne overstiger trykket i atrierne således, at AV-klapperne lukkes, så tilbagestrømning af blod til atrierne forhindres. Når ventriklerne har kontraheret sig så meget, at trykket i ventriklerne overstiger trykket i hovedpulsåren samt i lungepulsåren, åbnes aortaklappen og pulmonalklappen, og blodet strømmer ud i hovedpulsåren og lungepulsåren. Ventriklernes tryk falder igen til under atriernes tryk, hvilket påvirker, at AV-klapperne åbnes igen og hjertets cyklus starter forfra.\\
Hjertets cyklus igangsættes i sinusknuden ved aktionspotentialer, der føres til de forskellige dele af hjertet. Dette sker enten ved, at aktionspotentialet går fra hjertemuskelcelle til hjertemuskelcelle gennem åbne celleforbindelser, eller gennem åbne celleforbindelser mellem specialiserede hjertemuskelceller i hjertets specielle ledningssystem. Det specielle ledningssystem består af tre sammenhængende dele - AV-knuden, det hiske bundt gennem anulus fibrosus og det hiske bundt over i purkinjefibrene(se figur 3.3). \\
Hjertets ledningssystem har to hovedopgaver. Først at sørge for, at aktionspotentialet spredes hurtigt gennem hjertet, og dermed sørge for al hjertemuskulaturen i ventriklen kontraheres næsten samtidig. Denne næsten samtidige kontraktion medfører, at der inde i ventriklerne opbygges et effektivt tryk. Purkinjefibrene, som kun er i ventriklerne og ikke atrierne, gør at aktionspotentialerne spredes hurtigere i ventriklerne end i atrierne. Den anden hovedopgave er derfor at sikre en vis forsinkelse i impulsledning fra atrierne til ventriklerne. Forsinkelsen er mulig, da anulus fibrosus fungerer som en elektrisk isolator. Derfor skal aktionspotentialet ledes fra atrierne til ventriklerne via det specialiserede ledningssystem, og da AV-knuden leder aktionspotentialet særlig langsomt, opstår forsinkelsen. Dette medfører, at atriernes kontraktion fuldføres, før ventriklernes igangsættes, dermed er der sikret en tilstrækkelig fyldning af ventriklerne, før de pumper blodet videre. Denne spredning og udløsning af aktionspotentiale sker regelmæssigt, og er den afgørende faktor for hjertets kontraktions rytme.
\begin{figure}[htb]

	\centering	
	\includegraphics[width=1\textwidth]{Figurer/Snip20150410_6}
	\caption{Spredning af aktionspotentialer gennem hjertet \protect\footnotemark}
\end{figure}
\footnotetext{"Menneskets anatomi og fysiologi" s. 275 figur 9.9}

I figur 3.3 ses spredningen af aktionspotentialer gennem hjertet. Aktionspotentialet udløses i sinusknuden og forsinkes i AV-knuden. Dernæst ledes aktionspotentialet videre til ventrikelmuskulaturen. De farvelagte områder er de depolariserede områder og det ses, at atriernes depolarisering er afsluttet før ventriklernes er startet.

\section{Elektrokardiogram}
Et elektrokardiogram, EKG, afspejler hjertets elektriske aktivitet. Teknikken kaldes elektrokardiografi og udføres via elektroder, der er placeret forskellige steder på kroppen, primært omkring hjertet. Elektroderne måler den elektriske aktivitet via en overfladespænding, der går ud fra thorax. Det er disse elektriske spændinger, som danner de forskellige graf-udsving, som er EKG-signalets takker. Takkerne viser atriernes- og ventriklernes systole og diastole, og er inddelt i P-takken, QRS-komplekset og T-takken. Grafisk vil EKG signalet være vist som det ses på figur 3.4 nedenfor.

\begin{figure}[H]
	\centering
	\includegraphics[width=0.7\textwidth]{Figurer/Snip20150412_36}
	\caption{Normalt EKG-signal \protect\footnotemark}
\end{figure}
\footnotetext{http://en.wikipedia.org/wiki/QRS\_complex\#/media/File:SinusRhythmLabels.svg}

P-takken viser atriets depolarisering. Dernæst kommer PR-segmentet, som er den forsinkede strøm fra atrier til ventrikler. QRS-komplekset udgør ventrikeldepolariseringen. QRS-komplekset er større end P-takken, da muskelmassen i ventriklerne er større end atriernes muskelmasse, hvilket påvirker en højere elektrisk aktivitet. T-takken beskriver ventriklernes repolarisering. Denne er også mindre end QRS-komplekset, da repolariseringen forløber langsommere end depolariseringen.\\
Elektrokardiografi giver et billede af, hvordan hjertet fungerer. På figur 3.4 ses et EKG-signal for et raskt hjerte. Ved et raskt hjerte vil intervallerne mellem takkerne i EKG-signalet være følgende: 

\begin{itemize}
	\item PR Interval: 0.12 - 0.20 sekunder
	\item QRS Complex: 0.08 - 0.10 sekunder 
	\item QT Interval: 0.4 - 0.43 sekunder 
	\item RR Interval: 0.6 - 1.0 sekunder 
\end{itemize}

 Hvis hjertet ikke fungere optimalt, vil EKG-signalet se anderledes ud, og en sundhedsfaglig person vil kunne diagnosticere eventuelle sygdomme ud fra grafen.  En patient kunne eventuelt have atrieflimren, som er den sygdom, dette projekt omhandler.  


\section{Atrieflimren}
Atrieflimren forekommer, når atrierne ikke kontraherer sig ordentligt. Den hyppigste udløsning af atrieflimren forefalder pga. en serie af hurtige impulser (ekstrasystoler), hvilket er illustreret på første del af figur 3.5.  Ekstrasystolerne kommer fra den atriemuskulatur, som sidder nær lungevenerne i venstre atrium. Dermed bliver atriernes normale kontraktionsmønster ødelagt, og de begynder at "flimre". Under atrieflimren fungerer sinusknuden stadig som normalt, men har ingen kontakt til atrium.\\
Pga. arytmien mister man den regelmæssige atrietømning og får nedsat funktion af hjertets pumpning. Blodet vil ophobe sig i atriet og danne lokale tromber.  De kan løsrive sig og flyde med blodstrømmen ud i kroppen, hvor de kan sætte sig fast (embolisere). Ubehandlet emboliserende atrieflimren er årsagen til 1/3 af alle cerebral apopleksiske tilfælde. Derfor er det vigtigt at være opmærksom på tilstedeværelse af atrieflimren hos netop disse patienter.\\
Hvis arytmien står på i længere tid, og ventrikelfrekvensen er hurtig, kan det udløse hjerteinsufficiens med tiltagende dilatation og dårlig kontraktion af ventriklerne. \\
Atrieflimren opstår som anfald (paroksystisk), der spontant konverterer tilbage til normal sinusrytme efter få timer til dage. Med årerne bliver arytmien mere vedvarende (persisterende) for til sidst at blive kronisk.  De symptomer, som kan forbindes med atrieflimren er en øget træthed, åndenød og en forhøjet samt uregelmæssig puls, der kan være utydelig og hurtig. Desuden vil blodtrykket falde, og der kan være tegn på hjerteinsufficiens, både i højre og venstre side af hjertet. 

\begin{figure}[htb]
	\centering
	\includegraphics[width=1\textwidth]{Figurer/Snip20150412_32}
	\caption{Aktivitet i atrie ved atrieflimren\protect\footnotemark}
\end{figure}
\footnotetext{http://www.health.harvard.edu/heart-health/atrial-fibrillation-common-serious-treatable}

Diagnosen stilles via elektrokardiografi. EKG-grafen er domineret af mange irregulære og smalle QRS-komplekser uden ordentlige P-takker, som set på figur 3.5 ovenover. Med andre ord siges det, at der forekommer 220-300 små udsving/minut. Der kan også være forhøjede amplituder i frekvensspektret 300-400 Hz. Hvis disse forhold forekommer, kan patienten muligvis have atrieflimren.\\ 
Den hyppigste form for behandling er ved betablokkere, flekainid, dronaderon og amiodaron. Der indføres katere i venstre atrium, som ødelægger atriemuskulaturen, hvilket udløser flimren.

  