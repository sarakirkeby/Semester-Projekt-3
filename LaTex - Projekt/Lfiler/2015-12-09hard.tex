\section{Logbog}

\textbf{Dato:} 10-12-2015 \\
\textbf{Omhandler:} Hardware \\
\textbf{Ansvarlige:} Sara, Freja og Nicoline \\
\textbf{Logbog}

Dagsorden:
\begin{itemize}
	\item Justering af hardware
	\item Design og Implementering
	\item Transducer baggrund
	\item Test
\end{itemize}

Vi startede med at skrive delelementer af dokumentationen og rapporten ind i LaTex. En del af gruppen fortsatte med rapport skrivningen mens resten af gruppen gik i lab for at teste hardwaredelen med digitalt multimeter og en vandsøjle. Efter snak med Arne og Thomas og konsultation med det digitale multimeter måtte vi konkluderer at Analog Discovery har et lille offset som får indflydelse på målingerne af indgangssignalet. I følge Thomas kan brug af Analog Discovery have en uheldig side effekt hvis man måler på indgangssignalet med denne. Derfor måler vi ikke længere indgangssignalet.
Vi testede det fulde system med software i CAVE-lab, resultaterne var okay dog var der lidt problemer med diastolen. 
Senere udførte vi flere test af hardware delen og fik indsamlet flere målpunkter til dokumentationen af test. Vi fik vist at der er en lineær sammenhænge mellem væske tryk i vandsøjlen og den spænding der kommer på udgangssignalet som skal gå ind i DAQ'en.  