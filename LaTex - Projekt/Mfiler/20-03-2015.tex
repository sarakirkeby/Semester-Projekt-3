\chapter{Mødereferat}

\textbf{Dato: 20-03-2015} 

\textbf{Fremmødte:} Albert, Mohamed, Martin, Malene og Mads 

\textbf{Fraværende:} Lise, Cecilie og Sara

\textbf{Dagens dagsorden:}
\begin{itemize}
	\item Svar på spørgsmål omkring overgang fra forprojekt til hovedprojekt
\end{itemize}

\textbf{Referat: }\\ 
Design-mappe. Skal system test og acceptest ligge derinde?
De kan godt ligge der inde. \\
Skal vi teste ’Could’? 
MoSCoW kan bruges når man har mange use cases. Og kan bruges hvis der er vigtige dele der skal deles op i MoSCoW.
\\
Når man skriver det op i MosCow, burde der være en use case forbundet. 
\\
Skal vi sorterer nogle af dem fra (touch-skærm f.eks.)?
Den bør ikke være med i MoSCoW.
\\
Det der er ikke er funktionelle krav, skal ’bindes op’ på en use-case.
\\
Hvis DET er tidsmessigt af sammenhæng er det én usecase.
Hvis det ikke er tidsmessigt af sammenhæng er det to eller flere use cases.
\\
Vores EKG-system:
Aktør kontekst, kan vi lave en pil fra analog op til physionet?
Vi kan lave en ekstern aktør, som er physio-net.
\\
Hovedforløb i use-casen, hvordan skal det være i forhold til opkobling til patient/physionet
Som en forudsetning. 
Vi har et analog-signal, som skal opkobles til et EKG-signal.
\\
Aktør kontekst:
Vi skal have bruger i stedet for sundhedsprof.