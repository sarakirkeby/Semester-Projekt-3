\textbf{Dato:} 23-09-2015

\textbf{Fremmødte:} Jeppe, Freja, Mads, Tine og Nicoline

\textbf{Fraværende:} Sara

\textbf{Dagens dagsorden:}
\begin{itemize}
	\item Tilbagemelding på use cases, og diagrammer.
	\item Spørgsmål
	\subitem Skal use casene forholde sig til, hvad der kan testes i virkligheden eller er det i orden, at vi simulerer noget?
	\subitem Skal use casene beskrives helt præcist? Eks. stop-knappen skifter farve fra rød til grå.
	\subitem Hvad hedder "en blodtryks graf"?\\
\end{itemize}

\textbf{Referat:}
\\Der er fast vejledermøde tirsdag klokken 12:15.
\\Review med gruppe 3 planlægger og afholder vi selv.
\\Onsdag sender vi vores færdige materiale til Thomas, der kigger det igennem så vi har det klar til review med gruppe 3.
\\Vi har et LaTex dokument for hvert hovedafsnit, hvilket er ideelt for versionshistorik.
\\I forhold til kravspecifikation, ikke funktionelle krav mangler.
\\Undtagelse "Nulpunkts justeringen er ikke korrekt" hvad gør at den ikke er korrekt, kunne det være kalibreringen?
\\Tekniker i use case diagrammet, overvej hvorfor vi har valgt ham som primær aktør.
\\Vi skal måske uddybe, hvad er digitalt filter helt præcist laver. Laver en pænere graf. Kan hjælpe os til at finde sys- og diastolisk bodtryk.
\\Det digitale filter skal med i en use case, så skal kravene også indgå. Det vil være godt at uddybe kravene til filteret lidt mere.
\\Vi tester vores system ud fra, hvad der er muligt at teste i cave-lab. Alt andet kommer under perspektivering. 
\\Hvad hedder "en blodtryks graf"? Blodtrykskurve.



