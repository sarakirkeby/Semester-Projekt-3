\chapter{Mødereferat}

\textbf{Dato:} 23-09-2015

\textbf{Fremmødte:} Alle

\textbf{Fraværende:} Ingen

\textbf{Dagens dagsorden:}
\begin{itemize}
	\item Vores valg af institution
	\item Udkast til krav
	\item Spørgsmål
	\subitem Ansvarsområder i rapporten?
	\subitem Fast ugentlig vejledermøde tirsdag kl. 12.15
	\subitem Hvordan kalibrerer man i praksis?
	\subitem Er der forskel på kalibrering og en nulpunktsjustering?
	\subitem Hvilken vej peger pilene på use case diagrammer?
	\subitem Skal patient med i use case diagrammet, eller er han off-stage aktør?
\end{itemize}

\textbf{Referat:}
\\Valg af institution blev operationsstuen, hvilket er et godt valg. 
\\Udkast til krav, alarmen skal kunne tage højde for om patienten i forvejen har et kendt forhøjet blodtryk.
\\Der er forskel på kalibrering og nulpunktsjustering. Man kan overveje om man skal have en aftale om at kalibrerer apparatet en gang årligt, mens nulpunktsjusteringen skal foretages hver gang. Vi bør have en primær aktør, der er medicotekniker med egen use case. 
\\Skal man have en log in funktion? Det er normalt at operatøren giver sig selv til kende ved eks. ultralydsscanninger. Personale ID, dermed også login usecase. Man kunne lave en opstarts procedure der logger personalet ind og nulpunktsjusterer.
\\Lyder fornuftigt at have en gem funktion der gemmer alt det der er optaget efter en måling. Gem use casen skal hedde afslut som gemmer og logger ud.
\\Fast ugentligt vejledermøde tirsdag fra 13:15 hver uge. Udkast og dagsoden fremsendes før møde.
\\Thomas kigger på kalibrering til næste gang. Hvis en volt skal svarer til 1 mm kviksølv og det ikke stemmer overens kan man øge forstærkningen enten med hardwear eller softwear.Man kalibrerer et system for at finde ud af hvad "fejlen" på systemet er.
\\Der behøver ikke være pile på use case diagrammet, vi laver streger.
\\Patient er sekundær aktør.
\\I forhold til ansvarsområder på rapporten er det nødvendigt eller ej at skrive hvem der har lavet hvad? Det er ikke det mest kritiske. Det kommer alligevel på i versions historikken.
\\The not so short introduction to LaTex er et godt dokument at læse for at få lidt basisviden omkring programmet.
\\Når vi har funktionelle krav i vores use cases skal de fjernes fra MOSCOW krav.