\textbf{Dato:} 10-11-2015\\

\textbf{Fremmødte:} Jeppe, Freja, Nicoline, Sara og Mads\\

\textbf{Fraværende:} Tine\\

\textbf{Dagens dagsorden:}
\begin{itemize}
	\item Softwaredelen i forhold til kalibrering 
	\item Test af hardware
\end{itemize}

\textbf{Referat:}\\
Vi snakker med Samuel om vores software-diagrammer.\\
Softwaredelen skal stå for at der en en variabel der kan indstilles til brug ved kalibrering. Der er ikke tale 
om et ekstra program.\\
Hardware gruppen skal huske kalibrering i accepttesten.\\
Husk spændingsforsyningen i blokdiagrammet.\\
Brug spændingsregulator hvis man bruger batteri\\
Der kan godt være offset man ikkke kan komme ud over, skruer ned på funktionsgeneratoren så den i virkeligheden kommer ud med minus nogle få millivolt.\\
Når vi tester skal vi bruge en spændingsdeler eks. 100 gange forstærkningen, dele den med en spændingsdeler.\\
Tykke pinde til fumlebrættet!\\
Prøv med 1 forsærkning DC signal.\\ 
Husk jord i BDD\\
