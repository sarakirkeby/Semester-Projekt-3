\textbf{Dato:} 17-11-2015\\

\textbf{Fremmødte:} Jeppe, Freja, Nicoline, Sara, Tine og Mads\\

\textbf{Fraværende:} Ingen\\

\textbf{Dagens dagsorden:}
\begin{itemize}
	\item Diverse spørgsmål til HW 
	\item SW Brugergrænseflader
\end{itemize}

\textbf{Referat:}\\

Angående liner regression i Hardware; De måle usikkerheder der bliver skabt lige nu, er måleusikkerheden på indgangen af oscilloscopet. Det der kan gøres er at gå ned i lab i bunden og bruge det digitale osc. Der er en forskel på det vi gerne vil have, men dette kan skyldes måleusikkerhed.\\

Tidsfristen for print er 14 dage. Husk størrelse på komponenterne i multisim/ultiboard. \\

Angående software frekvens; Den metode Jeppe har gang i er rigtigt. Skal dog tjekkes for hvor robust det er. Den skal opdateres 5-10 sekunders mellemrum. Thomas vil have systemet til at opdatere systolen for hver top der kommer, sammen med de 2-3 der kom før. Alternativet er tre af gangen. Der er en der sidder og holder øje med det, når den er akut, så derfor behøver tallet måske ikke opdatere hvert sekundt. I ft alarmen, kan de dog være problematisk, hvis det først er ti sekunder efter, at man kan se at der er noget galt.\\

Man kan godt tage et rent BT og gange støj på for at få et testsignal som kommer ud.\\ 

Thomas ligger mest vægt på at vi laver sunde overvejelser i vores udvikling af systemerne. Husk at kunne argumentere for tingene. 